\documentclass[12pt]{article}
%%%%%%%%%%%%%%%%%%%%%%%%%%%%%%%%%%
\usepackage[pdftex]{graphicx}
\usepackage[usenames,dvipsnames]{color}
\usepackage{comment}
%%%% MIGHT TURN BACK ON WITH PXFONTS
%\usepackage{units,textcomp,xfrac}
%\usepackage{harvard}
\usepackage{amsmath, amsthm, amssymb}
\usepackage[colon]{natbib}
\usepackage{url}
\usepackage{fullpage}
\usepackage{bbm}
\usepackage{epstopdf}
\usepackage{setspace}
\usepackage{float}
\usepackage{tcolorbox}
\usepackage{verbatim}
\usepackage{pdfpages}
\usepackage{longtable}
\usepackage{lscape}
\usepackage{booktabs}
\usepackage{multirow}
\usepackage[english]{babel}
\usepackage{tabulary}
\usepackage{tabularx}
\usepackage{array}
\usepackage{sectsty}
%\usepackage{pdflscape}
\sectionfont{\large}
\usepackage{nicefrac}
\usepackage{rotating}
% \usepackage{floatrow} % added by MM for figure captions
% \usepackage{pxfonts}
% \usepackage{kpfonts}
\newcommand{\sups}[1]{\ensuremath{^{\textrm{#1}}}}
\newcommand{\subs}[1]{\ensuremath{_{\textrm{#1}}}}

\usepackage{needspace}
%\usepackage[subtle]{savetrees}

%%% MM command for rounding macro numbers %%%%%%%%%%%%%%
\usepackage{numprint} 
\nprounddigits{2}
\npdecimalsign{.}
\npthousandsep{,}
%%%%%%%%%%%%%%%%%%%%%%%%%%%%%%%%%%%%%%%%%%%%%%%%%%%%%%%%


\usepackage{subfig}
\usepackage{threeparttable}
%\usepackage{longtable}

% this allows you to use \sym from the stata-output tables
\newcommand{\sym}[1]{\rlap{#1}}

%%%%%%%%%%%%%%%%%%%%%%
%%%% APPENDIX STUFF
\usepackage[toc,page]{appendix}


%% COMMAND THAT ALLOWS YOU TO MAKE THE INDEPENDENCE SYMBOL
%% INDEPENDENCE SYMBOL ALSO REQUIRES GRAPHICX
\newcommand{\indep}{\rotatebox[origin=c]{90}{$\models$}}


\floatstyle{plain}

\newcommand\independent{\protect\mathpalette{\protect\independenT}{\perp}} % symbols-a4, p.106
\def\independenT#1#2{\mathrel{\rlap{$#1#2$}\mkern2mu{#1#2}}} 

\def\changemargin#1#2{\list{}{\rightmargin#2\leftmargin#1}\item[]}
\let\endchangemargin=\endlist 

\let\footnotesize=\small
\let\titlesize=\small

\usepackage{geometry}
% \onehalfspacing
\geometry{verbose,letterpaper,tmargin=1.0in,bmargin=1in,lmargin=1in,rmargin=1in}
%\geometry{verbose,letterpaper,tmargin=1.5in,bmargin=1.5in,lmargin=1.5in,rmargin=1.5in}

\renewcommand*{\thetable}{\arabic{table}}

% \usepackage[colorlinks]{hyperref}
\PassOptionsToPackage{hyphens}{url}\usepackage[colorlinks]{hyperref}
\hypersetup{
  colorlinks=true,
  citecolor=Blue,
  linkcolor=Red,
  urlcolor=Magenta}

\begin{document}
\DeclareGraphicsExtensions{.pdf,.tif, .jpg}

\title{ \vspace{-1.5cm}
Beyond Crime Rates: How Did Public Safety in U.S. Cities Change in 2020?\thanks{Author order is randomized. Massenkoff: maxim.massenkoff@nps.edu, Chalfin: achalfin@sas.upenn.edu. We thank David Abrams, Jacob Kaplan, John MacDonald, Evan Rose, Yotam Shem-Tov and seminar participants at the Claremont Graduate University for helpful feedback on an earlier version of this paper. Any remaining errors are our own.
} }
%Beyond Crime Rates: Measuring Changes in the \emph{Risk} of Crime Victimization
%Beyond Crime Rates: How Did the Risk of Crime Victimization Actually Change During the COVID-19 Pandemic?
\author{
  {\large Maxim Massenkoff} \\ {\normalsize Naval Postgraduate School}  \\ 
  \and {\large Aaron Chalfin} \\ {\normalsize University of Pennsylvania and NBER}  
  }

\date{\vspace{.5cm}This version: \today}
% \date{\vspace{.5cm} November 5, 2020}

\maketitle
\thispagestyle{empty} 

% \vspace{-1cm}
% \begin{center}
% \textbf{PRELIMINARY AND INCOMPLETE; PLEASE DO NOT CITE OR CIRCULATE WITHOUT PERMISSION}
% \end{center}

\begin{abstract}
\setlength{\baselineskip}{13pt} %temporary set single space

\vspace{0.0in} %insert vertical space (in inches); can be in cm, mm
% Source for leading cause: https://www.cdc.gov/injury/images/lc-charts/leading_causes_of_death_by_age_group_2018_1100w850h.jpg

\noindent
This paper argues that changes in human activity during the COVID-19 pandemic led to an unusual divergence between crime rates and victimization risk in US cities. Most violent crimes declined during the pandemic. But analysis using data on activity shows that the \textit{risk} of street crime victimization was elevated throughout 2020. People in public spaces were 15-30 percent more likely to be robbed or assaulted. This increase is unlikely to be explained by changes in crime reporting or selection into outdoor activities by potential victims. Traditional crime rates may present a misleading view of recent changes in public safety.


\end{abstract}


\noindent


\thispagestyle{empty}\baselineskip1.5\baselineskip\newpage{}



%%%%%%%%%%%%%%%%%%%%%%%%%%%%%%%%%%%%%%%%%%%%%%%%%%%%%%%%%%%%%%

%\tableofcontents

\newpage
\setcounter{page}{1}

\begin{tcolorbox}
\textbf{Significance statement}  \\ 
The onset of the COVID-19 pandemic brought massive disruptions to economic and social life, including an unprecedented spike in homicides. Overall crime, however, was down. We resolve this apparent paradox by showing that after accounting for the fact that people were spending more time indoors in 2020, the risk of victimization in public actually increased. These recent changes in crime and activity provide a stark illustration of how conventional crime rates can fail to capture changes in public safety.
%The crime rate decreased dramatically with the onset of the COVID-19 pandemic. However, we show that because people were staying indoors, the risk of victimization in public actually increased. These recent changes in crime and activity are a stark illustration of how conventional crime rates can fail to capture a notion of public safety.

\end{tcolorbox}

\section{Introduction}
%%%%%%%%%%%%%%%%%%%%%%%%%%%%%%%%%%%%%%%%%%%%%%%%%%%%%%%%%%%%%%
The onset of the COVID-19 pandemic brought massive disruptions to economic and social life, including an unprecedented spike in homicides \citep{cdc2020hom}. Overall crime, however, was down, spurring a conventional narrative that gun violence has been an exception \citep{graham2021atlantic, nass2020trace} and that, contrary to public opinion, the pandemic has not had a deleterious impact on public safety \citep{ashby2020initial,abrams2021covid,koerth2020many}. This understanding, based on analyses of traditional crime data, informs an important discussion among researchers, public safety advocates, and journalists about recent changes in crime. But is it substantively correct?  

This paper illustrates how changes in human activity during the COVID-19 pandemic led to an unusual divergence between official crime rates and victimization risk. We focus, in particular, on public violence---assaults and robberies occurring in public or commercial spaces---which are often perpetrated by strangers and which drive an outsize share of fear among the public \citep{ferraro1995fear, pickett2012reconsidering, hanslmaier2013crime, ferraro2017measurement}. Beginning in March 2020, public violence declined by approximately 35\% as people responded to disease risk and mandated lockdowns by spending more time at home. These offenses remained 10-15\% below 2019 levels throughout the summer before returning to 2019 levels in the fall. But what happened to the risk of violent crime that people faced in while spending time in public spaces? Because our principal measure of public safety---the crime rate---is so limited, conventional analyses are unable to answer this question.\footnote{From its earliest inception, the per capita crime rate has always been intended to provide a rough proxy for victimization risk \citep{boggs1965urban, stipak1988alternatives, vaughan2020promise, ramos2021improving}. %Prior research has proposed various of ways of moving beyond crime rates to better capture victimization risk. 
We provide a brief history of prior thinking about how to compute an ``activity adjusted crime rate" in Appendix \ref{sec:appendixA}.}

%which is arguably one of the most fundamental questions about how urban life has changed during the pandemic.

\paragraph{Research strategy and key findings}
Using both administrative and survey data, as well as information on the movement of people based on mobile location data, we study the risk of public victimization in 2020. Newly-available survey data from the American Time Use Survey as well as foot traffic data from Google, Apple, Facebook and Safegraph, show that time spent in public spaces fell markedly in March 2020 before slowly rebounding thereafter. While Americans eventually began to return to many of their normal activities, even at the end of 2020, time spent outdoors remained about 20\% lower than in 2019.

We then compare the number of violent crimes to measures of time spent away from home in order to calculate the risk of street crime victimization. We formulate this as the count of violent crimes occurring in public spaces divided by a measure of time spent in public places, derived from the American Time Use Survey and mobile location-based data from SafeGraph. While analyses of traditional crime data show discrete drops in offending during the post-pandemic lockdown period, we find that, in 2020, the risk of outdoor street crimes initially \emph{rose} by more than 40\% and was consistently between 10-25\% higher than it had been in 2019 through the remainder of the year.\footnote{For analyses of traditional crime data during the COVID-19 pandemic, see \citet{ashby2020initial, boman2020has, gerell2020minor, hodgkinson2020show, langton2021six, payne2020covid, piquero2020staying, shayegh2020staying, abrams2021covid, boman2021global, campedelli2021exploring, de2021druglords, nivette2021global}.}$^{,}$\footnote{These findings complement and greatly extend recent research which uses Google's COVID-19 Community Mobility Reports to assess post-pandemic changes in criminal opportunities in the United States \citep{lopez2021crime} and the United Kingdom \citep{halford2020crime}. While this research notes that criminal opportunities shifted in 2020, it does not utilize a pre-pandemic measure of mobility, does not distinguish between crimes that occurred in residential and public locations, does not address selection effects and ultimately, it does not construct a quantitative test of the change in victimization risk during the pandemic.}

The results are consistent across two independent analyses. The first analysis utilizes administrative crime data and location-based mobility data from Safegraph for New York City, Los Angeles and Chicago, the three largest cities in the United States. The second analysis relies entirely on national survey data from the National Crime Victimization Survey and the American Time Use Survey. Critically, the administrative data and the survey data point to the same conclusion: that Americans were more likely to be robbed or assaulted while spending time in public after March 2020 than they had been previously. 

%One major concern is selection. If the composition of people who spent time outdoors shifted in 2020 so that people in public had higher ex ante risk of victimization, we would be overstating the increase in risk. In Appendix \ref{sec:appendix_alternative}, we show that changes in activity do not imply differential changes in risk and that, contrary to the concern, crime victims in 2020 had on average \textit{lower} past rates of victimization based on demographics. A second threat is reporting, but data from the NCVS suggests that 2020 saw no unique changes to the reporting rate.

We consider two main challenges to this conclusion. First, it is possible that the reporting of crimes to law enforcement changed in 2020. However, survey data shows no dramatic changes in the propensity of victims to report crime (see Appendix \ref{sec:reporting}). A second notable threat is selection. If in 2020, people selecting into outdoor activities---and who were thus at risk of a street crime---were those with a higher baseline risk of victimization, our activity-adjusted estimates of crime risk could overestimate the change in risk. We show in Appendix \ref{sec:selection} that compositional shifts among people in public imply no large changes in crime, and that 2020 victims were, if anything, at lower risk of victimization historically. This suggests that selection effects do not bias us toward a finding of increased risk.

%%%%%%%%%%%%%%%%%%%%%%%%%%%%%%%%%%%%%%%%%%%%%%%%%%%%%%%%%%%%%%
\section{Results}
%%%%%%%%%%%%%%%%%%%%%%%%%%%%%%%%%%%%%%%%%%%%%%%%%%%%%%%%%%%%%%
\subsection{Mobility}
Data on time spent in public spaces in 2019 and 2020 are presented in \autoref{fig:city_mobility_combined}. The figure presents trends using mobility data from Safegraph (panel A), Google (panel B), Apple (panel C), Facebook (panel D) as well as ATUS survey data (panel E). %\footnote{While all four mobility datasets are available in 2020, only the Safegraph data are available earlier, thus allowing us to observe changes in activity that account for seasonal variation. } 
For the mobility data, we restrict the sample to our three cities---New York, Los Angeles and Chicago---and aggregate across the cities to present information for the combined sample.\footnote{We present the same data, disaggregated by city in Appendix \ref{sec:appendix2} and note that patterns are extraordinarily similar across cities. While the time-path of the pandemic's impact differed widely among the three cities, the fact that mobility patterns are broadly similar suggests that these are driven to a greater extent by lockdowns or national trends than by localized health risks. }

All five sources of data indicate that outdoor foot traffic declined steeply at the beginning of the pandemic. In the Safegraph data, the total count of unique neighborhood visits across our three cities fell from an average of 220 million to 110 million from 2019 to 2020, a 50\% decrease.\footnote{The Google, Apple and Facebook data are more limited in that they are only available for 2020 and provided in terms of indexes relative to early 2020. Apple data is since January 13, 2020. The Google baseline is Jan 3–Feb 6, 2020 \citep{google_data}. Facebook baseline is from the end of February 2020 \citep{facebook_documentation}. Still, these data sources support a similar story.} The indexes in panels B, C and D suggest similar decreases. Apart from grocery shopping, the Google indexes in panel B decline by 60\% and remain 20-50\% lower throughout the year. The Apple travel indexes in panel C  fall by 60-80\% in April and show a sharper rebound---likely due to their indexing to a winter month. The Facebook index in panel D shows a sustained 20\% increase in staying put, a measure of the fraction of people staying in a small area for an entire day.

Finally, we consider survey data from the ATUS in panel E, plotting the share of waking hours spent away from one's home by month in 2019 and 2020. In 2019, respondents reported spending approximately 7 of their waking hours away from home, a statistic which exhibits relatively little seasonal variation. By May 2020, this had fallen to 3 hours.\footnote{The ATUS was temporarily suspended between mid-March and mid-May 2020 due to pandemic-induced lockdowns.} Throughout the rest of 2020, time spent outside remained about 30\% lower than in the previous year. Though the precise magnitudes differ somewhat depending on the source of data used, the broad trends are similar.

\subsection{Victimization Risk}
We plot crimes known to law enforcement for our combined New York City-Los Angeles-Chicago sample in panel A of \autoref{fig:city_crime_combined}, presenting monthly counts of violent crimes separately for 2019 and 2020.\footnote{Results for individual cities can be found in Appendix \ref{sec:appendix2}. %An alternative set of results using a narrower definition of public locations is available in \autoref{fig:city_crime_alt_combined}.
} Public violence was approximately 10\% higher in the winter of 2020 than the winter of 2019 but began to fall steeply after March 2020. In April 2020, public violence was approximately one third lower than it had been in April 2019. Consistent with the seasonality documented in prior research \citep{andresen2013crime, jacob2007dynamics}, public violence rose during the spring and summer months in both years.  However, throughout the summer and into the fall months, these crimes remained between 5-20\% lower in 2020 than they had been in 2019. Aggregating across the entire pandemic period of 2020 (March through December), public violence was 19\% lower in 2020 than it had been in 2019.\footnote{A corresponding decline in violent victimization is also observed in survey data from the NCVS which recorded a 24\% decline between the 2019 and 2020 surveys.}

Panel B plots violent street crime risk per 100,000 people using Safegraph data to measure the change in activity. In March 2020, street crime victimization risk was roughly equal to risk in the previous year at approximately 5 violent street crimes per month per 100,000 person-visits to a community. %\footnote{This statistic cannot be directly compared to per capita crime measures generated from administrative data as the denominator uses a different proxy for exposure.} 
However, at the same time that street crimes fell by more than 30\% in April 2020, the \emph{risk} of street crime victimization rose by nearly 40\%. Victimization risk remained elevated throughout the summer---approximately 12\% higher in July 2020 than July 2019---and never fully returned to trend, even by the end of the year. For the March-December period, street crime victimization risk was approximately 14\% higher in 2020 than it had been in 2019, despite a 19\% decline in the incidence of public violence. 

Next, we present the same analysis using national survey data from the NCVS and the ATUS. Figure \ref{fig:ncvs_nat_ests}, Panel A plots the total number of outdoor violent crimes in the United States, using NCVS respondents living in metropolitan areas. The number of outoor violent crimes fell from approximately 1.6 million in 2019 to 1.3 million in 2020, a 17\% reduction. Panel B plots the number of violent victimizations occurring in public spaces divided by the number of hours spent in public spaces reported in the ATUS. While the number of crimes fell by 17\%, the \emph{risk} of public victimization rose by approximately 25\% in 2020---from 9 to  12 victimizations per million hours spent in public spaces. This analysis suggests that the increase in crime risk reported in \autoref{fig:city_crime_combined} is not unique to the three largest cities in the US, and provides further assurance that the city-specific findings are unlikely to be an artifact of the limitations of Safegraph mobility data. 

%Taken together, these results suggest that the increase in risk was not restricted to an aberration in gun violence. While gun violence rose by 47\% in the three largest cities in the United States and by around 30\% nationally, the risk of violent street crimes generally rose as well---by around 25\% nationally. %Though street crimes appear to be slightly less sensitive to the effects of the pandemic than shootings, the trends are much more similar after accounting for changes in criminal opportunities.

\subsection{Robustness and Heterogeneity}
In Appendix \ref{sec:appendix1} we probe the robustness of our main results. We briefly summarize these analyses here. First, we reproduce \autoref{fig:city_crime_combined} using ATUS survey data to measure the change in mobility between 2019 and 2020 (Appendix Figure \ref{fig:atus_risk}). %\footnote{The ATUS data does not contain the city of residence, so we restrict to respondents living in a metropolitan area to approximate the mobility changes in our three-city sample.}
The estimate is even larger, suggesting a 32\% increase in risk over the sample period. We consider an alternative definition of public violence, limiting the analysis to crimes that occur in outdoor locations as opposed to all non-residential locations and find similar results (Appendix Figure \ref{fig:city_crime_alt_combined}). We also consider whether the types of activities that people are engaged in while outside their homes may have shifted in 2020 towards activities that carry a higher level of inherent risk (Appendix Figure \ref{fig:atus_where}). While the amount of time spent in public decreased markedly in 2020, the proportion of time spent on each of several different activities---including walking, driving or at work---did not change appreciably. 

%As an additional check on our findings, we merge data on twenty-five large U.S. cities collected by \citet{abrams2021covid} with Google mobility data for those cities and plot the change in violent crimes between 2019 and 2020 against the change in mobility from Google's baseline period (Jan 3rd–Feb 6th, 2020) to the post-pandemic period. We present those data in Appendix Figure \autoref{fig:abrams_google_all_cities}. Consistent with our main analysis, almost all cities lie above the 45-degree line, implying that the drop in activity was larger than the drop in crime. This suggests that risk, overall, increased. 

Finally, we consider two main challenges to our conclusion that the risk of street crime victimization rose in 2020 (Appendix \ref{sec:appendix_alternative}). First, it is possible that the reporting of crimes to law enforcement changed in 2020. But comparing the pooled 2015-2019 and 2020 waves of the NCVS, we see little evidence that crime reporting changed appreciably in 2020. 

A second potential issue is selection. If in 2020, people selecting into outdoor activities---and who were thus at risk of a street crime---were those with a higher baseline risk of victimization, our activity-adjusted estimates of crime risk could be biased towards a finding of increased risk. We assess the importance of selection in three different ways (Appendix Section \ref{sec:selection}). First, in a reweighting exercise \citep{dinardo1996labor}, we use the ATUS and NCVS to calculate a counterfactual 2020 crime rate using observed activity changes and 2015-2019 victimization rates by age, race, gender and household income. We find that the counterfactual rates are nearly identical to actual crime, suggesting that selection into activity among higher-risk groups is not an important source of bias. Second, using microdata from New York City and Los Angeles, we directly examine changes in the demography of crime victims. If anything, crime victims in the police data had \textit{lower}, not higher, historical rates of victimization. Third, to test whether the increase in risk is driven by people in high-risk neighborhoods spending relatively more time outside, we condition on Census block group fixed effects, finding that within-neighborhood risk also increased during the pandemic. These analyses support our main finding that risk for the average person increased.

%%%%%%%%%%%%%%%%%%%%%%%%%%%%%%%%%%%%%%%%%%%%%%%%%%%%%%%%%%%%%%
\section{Materials and Methods}
%%%%%%%%%%%%%%%%%%%%%%%%%%%%%%%%%%%%%%%%%%%%%%%%%%%%%%%%%%%%%%
We study changes in victimization risk using two different analyses. We begin with an analysis of administrative crime and mobility data from the three largest cities in the United States, computing an activity-adjusted measure of the risk of public violence. Next, we perform the same analysis using survey data from the National Crime Victimization Survey and the American Time Use Survey. In this section, we describe each of these data sources.

\subsection{Victimization Data}
There are two principal measures of crime in the United States. The first measure is derived from administrative data on crimes known to law enforcement. These data are compiled annually by the Federal Bureau of Investigation's Uniform Crime Reporting program and are increasingly made available directly to researchers by city planners in large US cities. The second measure of crime is derived from survey data from the National Crime Victimization Survey, an annual survey of a representative cross-section of American households. %We use each of these data sources, in turn.

\subsubsection{Administrative Crime data}
For our city-level analysis, crime is measured using publicly available microdata from New York City \citep{nyc_crime_data}, Los Angeles \citep{la_crime_data} and Chicago \citep{chicago_crime_data}. In each city, the microdata represent the universe of crimes known to law enforcement. For each crime, we observe the date, time, and type of crime. In New York City and Los Angeles, we also observe the age, gender and race of crime victims. %We use these data to address the possibility that the demographics of crime victims changed in 2020.

Critically, in all three cities we observe detailed information on the location of each crime, allowing us to identify whether the crime occurred inside a residence or in a public setting. In our primary analysis, we consider a crime to occur in a public space if it happened in any location that is not within the confines of a residential building or a connected property such as a yard or garage. Using this definition, public spaces include streets, parks and alleyways as well as commercial establishments and offices where people work. We also present alternative results where we count only streets, parks and alleyways as public spaces and confirm that the findings remain substantively unchanged (Appendix Section \ref{sec:alt_public}).   

\subsubsection{National Crime Victimization Survey}
Our national analysis measures crime victimization using data from the National Crime Victimization Survey (NCVS), a nationally representative repeat cross-sectional survey of American residents. The NCVS is widely regarded as the most accurate source of data on crime victimization and the reporting of crimes to law enforcement in the United States \citep{gutierrez2017silence}. The survey directs respondents to indicate if they have been the victim of a crime during the past six months. If so, respondents are asked to identify the types of crimes they have experienced and provide key details about each incident including the type of location in which the crime occurred. We use this information to focus on the subset of violent crimes that occurred in public spaces. Unlike the administrative crime sources, the survey data do not indicate the date or month of the victimization, so these analyses are limited to the annual level. For details about how we classify crimes as occurring in either public or residential locations in both the administrative data and the NCVS, see Appendix \ref{sec:safegraph_deets}, section 1.

Because rates of crime reporting to law enforcement may have changed in 2020, we present an additional analysis of reporting behavior using the 2015-2020 waves of the NCVS. Focusing on the subset of individuals living in metropolitan areas, we test whether victims became either more or less likely to report violent victimizations to law enforcement in 2020.

\subsection{Mobility data}
Our next task is to measure how the availability of potential street crime victims changed during the pandemic. In an ideal world we would observe the number of person-hours spent in public spaces and use this as a denominator against which to compare street crimes in 2019 and 2020. While activity cannot be captured perfectly using available data, we employ several data sources that allow us to construct proxy measures. In this section, we describe each of these data sources. 

% \footnote{The data have been used in several academic papers, listed \href{https://www.safegraph.com/publications/academic-research}{here}. Most applications are related to research on the COVID-19 pandemic, although see \citet{chen2019racial}.}

\subsubsection{Cell phone mobility data}
For our analysis of crime in New York City, Los Angeles and Chicago, our primary data source is SafeGraph's Neighborhood Patterns data, anonymized GPS data from cell phones with location services enabled \citep{sg_neighb}. These data provide a count of the number of unique people present in each Census block group in each month, disaggregated according to some broad features such as the Census block group that each visitor came from and whether a given day was a weekday versus a weekend day. %\footnote{A Census block group is a small enumeration area meant to contain 600-3,000 residents.} 
There are two distinct advantages of the Safegraph data.  First, the data provide a direct measurement of the movement of people throughout the day that does not rely on self-reports from survey data.  Second, microdata are available by neighborhood, which allows for city-level and even sub-city level analyses. 

% On this point, we note that SafeGraph has explored the potential selection bias of tracked users by comparing their geography, education, and household income to Census data, finding a high correlation, implying that the sample of users is representative of the population at the census block group level \citep{safegraph_bias}.
While the data allow us to identify foot traffic with considerable granularity, they are subject to limitations. In particular, the data come from a non-random sample of users using certain apps. If this sample exhibits differential trends in activity, we could mismeasure changes in risk. We test the reliability of the data by measuring its trends against four other known sources of mobility information. In addition, to account for the fact that the data is a sample all visitor counts are scaled by the inverse sampling probability in the home Census block group. More details about the data construction are in the Appendix \ref{sec:safegraph_deets}, section 2. 
 
As a check on the SafeGraph trends, we also use geolocation data gathered by \citet{google_data}, \citet{apple_data}, and \citet{facebook_data}.\footnote{These datasets have been used in several studies on COVID-19 and social distancing \citep[e.g.][]{cot2021mining, venter2020covid}; examples which emphasize methodology are \citet{ilin2021public} and \citet{arnal2020private}.} Each of the three sources uses a different technique to define mobility. The Facebook data consists of two daily indexes, provided at the county level, measuring how often people stayed in a given Bing tile (a 600 square meter area) and how many different tiles people visited \citep{facebook_mobility}. The Apple data measures direction requests for walking, driving, and transit compared to a baseline measured on January 13, 2020 \citep{cot2021mining}. The Google data show mobility trends across four different categories: grocery and pharmacy, parks, retail and recreation, residential, and workplaces \citep{google_mobility, cot2021mining}. The Apple data is based on all Apple Maps users, while Google and Facebook data is based on those with location services enabled. 

\subsubsection{The American Time Use Survey}
In our national analysis, we measure changes in time spent in public spaces using survey data from the American Time Use Survey (ATUS). The ATUS is designed to study how and where people living in the United States spend their waking hours. The survey has been administered annually to a nationally representative sample of between 20,000 and 40,000 American households since 2003 and asks respondents to carefully document how they spent their time during a recent day. Importantly, the data permit researchers to delineate between time spent at one's home versus elsewhere. Respondents are surveyed throughout the year except for a gap corresponding to April 2020. We use each individual's survey date to build a monthly repeated cross-section of time use, limiting the data to residents in metropolitan areas. Our measure of outdoor activity comes from aggregating time spent in all activities occurring outside of a home or yard.\footnote{While the ATUS does not ask respondents where they slept, it does record detailed data about where a respondent spent his or her waking hours.} 

\subsection{Methods}
The crime rate is the count of crimes divided by a measure of the population at risk. Our crime measure is the number of violent street crimes---assaults and robberies occurring in public---measured either using local administrative data or national survey data. The ideal denominator would be a direct measure of the amount of time individuals spent outdoors, so that a crime rate could be calculated in terms of person-hours spent in public: 
\begin{equation}
Risk_t = \frac{Crimes_t}{PersonHoursOutside_t}
\end{equation}

For our national analysis, we estimate the denominator in (1) using ATUS survey data, which allows us to directly compute a national estimate of the number of person-hours spent in public spaces in each year. For our analysis of victimization risk in New York City, Los Angeles and Chicago, we measure the denominator using data from Safegraph. We focus on this mobility data source because it is available in both 2019 and 2020 and its measures have a direct interpretation.


The proxy that Safegraph provides for the denominator in (1) is a measure is of the total number of visitors across neighborhoods, counting both residents and non-residents. This will increase by 1 for a given Census block group whenever a new person enters the area and remains there for at least a few minutes. The SafeGraph data are thus accurate proxies for our ideal denominator if overall time spent outdoors scales linearly with the number of unique census block groups visited. For instance, if Census block groups in New York City went from 15,000 unique daily visitors in 2019 to 10,000 unique daily visitors in 2020, this would register as a 33\% decline in outdoor activity in our data. However, this would overstate the decrease if people were spending the same amount of time outside but in a fewer number of neighborhoods, and it would be an underestimate if people were spending even less time outside when they were in their home neighborhoods.The alignment we see between SafeGraph and the other activity measures provides some assurance that idiosyncracies in SafeGraph's data do not drive our results.




%%%%%%%%%%%%%%%%%%%%%%%%%%%%%%%%%%%%%%%%%%%%%%%%%%%%%%%%%%%%%%
\section{Discussion}
%%%%%%%%%%%%%%%%%%%%%%%%%%%%%%%%%%%%%%%%%%%%%%%%%%%%%%%%%%%%%%
Beginning in March 2020, the number of street crimes in America's cities declined sharply as people adapted to disease risk and mandated lockdowns by spending more time at home. In contrast to analyses of traditional crime data \citep{abrams2021covid}, we find that, in 2020, the risk of outdoor street crimes initially \emph{rose} by more than 40\% and was between 15\% and 30\% \emph{higher} than it had been in 2019 through the remainder of the year. These differences are unlikely to be a mechanical artifact of changes in crime reporting or differential selection into outdoor activity during the COVID-19 pandemic. 
The results suggest that the decline in overall crime in 2020 is an artifact of changes in criminal opportunities and that worrying changes in public safety were not restricted to a spike in homicides. %In contrast to the narrative in the popular media \citep{koerth2020many, nass2020trace}, the increase in homicides was not an aberration. 
These shifts mirror other troubling increases in motor vehicle fatalities \citep{stewart2022overview} and drug overdoses \citep{stephenson2020drug}.

%  Analyses of standard administrative crime data could present a misleading view of what has happened to public safety since the beginning of the COVID-19 pandemic.

Per capita crime rates are used to capture a notion of risk. But, as we argue, this shorthand became less reliable with the onset of the COVID-19 pandemic. The data and methods we employ suggest a path forward in going beyond crime rates to more accurately measure risk, a long-time goal of crime researchers that is more attainable now due to recent advances in mobility data. By comparing public victimization to mobility, we can better understand how public safety varies across cities, among neighborhoods within cities, by time of day, or within demographic groups. Better measurement can lead to improvements in policy, for example through more cost-effective deployment of police officers, social services, and public infrastructure investments such as street lighting.

%---for example, more cost-effective deployment of police officers and social services and the enhanced targeting of investments in public infrastructure such as street lighting, garbage removal and the remediation of blighted properties. 

\newpage
\singlespacing
\bibliographystyle{chicago}
\bibliography{bib}


\input{figs}
\newpage 
\clearpage 
\input{tables}

\counterwithin{figure}{section}
%\counterwithin{table}{section}

\newpage 
\clearpage
\vspace*{90mm} \\
\begin{center}
\Huge{\textbf{Online Appendix}}
\end{center}

\newpage
\newpage
\clearpage
\appendix



\singlespacing
%%%%%%%%%%%%%%%%%%%%%%%%%%%%%%%%%%%%%%%%%%%%%%%%%%%%%%%%%%%%%%
%%%%%%%%%%%%%%%%%%%%%%%%%%%%%%%%%%%%%%%%%%%%%%%%%%%%%%%%%%%%%%
%%%%%%%%%%%%%%%%%%%%%%%%%%%%%%%%%%%%%%%%%%%%%%%%%%%%%%%%%%%%%%
%%%%%%%%%%%%%%%%%%%%%%%%%%%%%%%%%%%%%%%%%%%%%%%%%%%%%%%%%%%%%%
\section{History of Measuring Activity-Adjusted Crime Risk} \label{sec:appendixA}
%%%%%%%%%%%%%%%%%%%%%%%%%%%%%%%%%%%%%%%%%%%%%%%%%%%%%%%%%%%%%%
%%%%%%%%%%%%%%%%%%%%%%%%%%%%%%%%%%%%%%%%%%%%%%%%%%%%%%%%%%%%%%
%%%%%%%%%%%%%%%%%%%%%%%%%%%%%%%%%%%%%%%%%%%%%%%%%%%%%%%%%%%%%%
%%%%%%%%%%%%%%%%%%%%%%%%%%%%%%%%%%%%%%%%%%%%%%%%%%%%%%%%%%%%%%
Researchers typically track changes in public safety using the crime rate: the number of crimes known to law enforcement divided by an area's population \citep{nolan2004establishing}. Crime rates are used to compare criminal activity across cities and to understand how public safety has changed over time. The advantage of the crime rate is that it is transparent and can be straightforwardly computed for individual cities as well as nationally using publicly-available data. 

The crime rate is a convenient albeit imperfect heuristic for public safety, a core concept but one that can be difficult to define and even harder to measure. While public safety can refer to a number of different ideas, a common conception employed in research and the policy world involves the risk of victimization for a typical citizen \citep{boggs1965urban, stipak1988alternatives, vaughan2020promise, ramos2021improving}. Victimization risk motivates the central statistics --- crime incidence and prevalence --- that are released by the U.S. Bureau of Justice Statistics in their annual distillation of the U.S. National Crime Victimization Survey, the sole national victimization survey in the United States. Victimization risk is likewise a key ingredient in how members of the public think about safety, especially when it comes to the risk of becoming the victim of a street crime \citep{ferraro1995fear, pickett2012reconsidering}.

Using the crime rate as a proxy for public safety has led to a litany of critiques in the criminology literature. The most common criticism of the crime rate is that crimes that become known to law enforcement --- the only city-level measure of victimization that is consistently available in the United States --- represent only a subset of criminal activity. In particular, researchers have worried about the ``dark figure of crime," the number of crimes which are unreported to and undetected by state or local police agencies \citep{biderman1967exploring, penney2014dark}. In the presence of victim underreporting and incomplete crime detection by police, the crime rate will underestimate the true risk of becoming a crime victim, a problem which is thought to be particularly large for stigmatized crimes like domestic violence and sexual assault \citep{felson2005reporting}, crimes with lower social costs and relatively low clearance rates like theft \citep{skogan1977dimensions} and crimes against juveniles who may be particularly apprehensive about interacting with the police \citep{finkelhor2001factors}.%
\footnote{In the United States, a related problem is that, despite national reporting standards set forth by the Federal Bureau of Investigation, inconsistent recording practices by local law enforcement agencies can distort the validity of between-city crime comparisons.} 
While the difficulty of accurately recording crime represents an important challenge to using the crime rate, this challenge is far from insurmountable. Using national survey data, it is possible to estimate the magnitude of the dark figure of crime, including hard-to-measure crimes like domestic violence \citep{bachman1994violence} or crimes against juveniles \citep{hashima1999violent}. Likewise, by focusing on crimes which tend to be consistently recorded --- for example, murder and motor vehicle theft --- measurement artifacts can be minimized.\footnote{Other challenges to using the crime rate to understand the risk of victimization include those created by population heterogeneity and differential selection into risky activities.}

In this paper, we focus on a deeper and more challenging issue in the measurement public safety --- a problem that has been noted by researchers for many years but which, due to severe data constraints remain unresolved. While measuring the crime rate's numerator has received the lion's share of scholarly attention, it is actually the fraction's denominator that poses the most salient challenge for researchers \citep{boggs1965urban}. That is, assuming that we have an accurate accounting of the number of crimes, against what reference group should that number be compared for measuring public safety? While population is a useful starting point, for several reasons it may be a poor proxy for the number of criminal opportunities. 

As has been recognized in previous research, a city's resident population is an imperfect proxy for the number of individuals who are at risk to become a crime victim \citep{boggs1965urban, boydell1969demographic, skogan1978victimization, stipak1988alternatives, newton2018macro,gerell2021does, vaughan2020promise, ramos2021improving}, particularly the victim of the types of street crimes that capture an outsize amount of fear in the public's imagination \citep{skogan1986fear, ferraro1987measurement}. For example, prior to the COVID-19 pandemic, Manhattan Island, home to approximately 1.7 million individuals, swelled to a population of approximately 3.4 million people during a typical workday. Clearly, deflating the number of crimes by 1.7 million leads to a crime rate that is biased upwards, an issue noted by \citeauthor{boggs1965urban} nearly sixty years ago.\footnote{As Boggs noted, ``spuriously high crime occurrence rates are computed for central business districts, which contain small numbers of residents but large numbers of such targets as merchandise on display, untended parked cars on lots, people on the streets, money in circulation, and the like.''}
At the same time, since people who work in Manhattan likely spend less time there, on average, than Manhattan residents, using 3.4 million would yield an underestimate. Ideally, we would deflate crime by the number of person-hours spent in Manhattan during a given period. However this figure is, for obvious reasons, difficult to estimate.

A related concern is that crime risk is, in large part, a function of its opportunity, an idea that criminologists generally refer to as routine activities theory \citep{cohen1979social, felson1987routine, branic2015routine}. This theory offers a parsimonious explanation for why juvenile offending peaks after school lets out \citep{fox1997after, fischer2018juvenile} and why vehicle thefts tend to be counter-cyclical, falling during recessions --- there are fewer vehicles to steal from downtown areas \citep{cook1985crime,bushway2012overall}. If a population changes its behavior due to an exogenous shock, the crime rate might change even if the risk of crime for those who are unresponsive to the shock remains constant. This issue will be especially pronounced in times where activity fluctuates; sharp changes in the crime rate could belie less radical changes in risk or vice versa. 

Prior literature has proposed several different innovations to better measure the risk of crime. Recognizing that the risk of exposure varies from crime to crime and might be poorly correlated with population, the earliest literature proposed a series of crime-specific denominators including vehicles registered for vehicle thefts \citep{lottier1938distribution, cohen1985risk}, female population for rape \citep{boggs1965urban} and the number of occupied housing units as a denominator for residential
burglary \citep{boggs1965urban, minnesota1977crime}. Recognizing the multi-dimensional nature of risk, others have proposed regression adjustment as a way to empirically model crime risk across population as well as other relevant denominators such as vehicles registered \citep{stipak1988alternatives}. 

Subsequent research has focused on time spent at and away from home as creating a salient measure of the risk of exposure. Starting with a seminal contribution by \citet{cohen1979social} which noted that individuals spend, on average, two-thirds of their time at home, one hour per day outdoors, and the remaining time indoors at a location that is not their own residence. Using these denominators, the authors calculated that, per unit of time, the risk of assault by a stranger on the street was more than 20 times greater than the risk of assault at home by someone who is known to the victim. While this result may seem perfectly intuitive today, at the time, this research was instrumental in overturning a common belief by scholars that home was the place where crime risk was greatest \citep{vaughan2020promise}. A recent innovation proposed by \citet{vaughan2020promise} and \citet{lemieux2012risk} is to use national survey data to measure changes in the time that individuals spend outside their homes. By comparing data on victimization from the National Crime Victimization Survey to data on time use from the ATUS, these papers have computed victimization rates for various activities, finding that victimization risk for the 2003-2008 period is approximately 8 per one million hours spent in public. We update and build upon this approach, augmenting national survey data from the ATUS with new data from mobile phone-based geo-location software.

\newpage 
%%%%%%%%%%%%%%%%%%%%%%%%%%%%%%%%%%%%%%%%%%%%%%%%%%%%%%%%%%%%%%
%%%%%%%%%%%%%%%%%%%%%%%%%%%%%%%%%%%%%%%%%%%%%%%%%%%%%%%%%%%%%%
%%%%%%%%%%%%%%%%%%%%%%%%%%%%%%%%%%%%%%%%%%%%%%%%%%%%%%%%%%%%%%
%%%%%%%%%%%%%%%%%%%%%%%%%%%%%%%%%%%%%%%%%%%%%%%%%%%%%%%%%%%%%%
\section{Data Details} \label{sec:safegraph_deets}
%%%%%%%%%%%%%%%%%%%%%%%%%%%%%%%%%%%%%%%%%%%%%%%%%%%%%%%%%%%%%%
%%%%%%%%%%%%%%%%%%%%%%%%%%%%%%%%%%%%%%%%%%%%%%%%%%%%%%%%%%%%%%
%%%%%%%%%%%%%%%%%%%%%%%%%%%%%%%%%%%%%%%%%%%%%%%%%%%%%%%%%%%%%%
%%%%%%%%%%%%%%%%%%%%%%%%%%%%%%%%%%%%%%%%%%%%%%%%%%%%%%%%%%%%%%
In this appendix we provide additional detail on the sources of data used in the paper and how our measures of crime and mobility were constructed from those data. 

\subsection{Defining Crime Locations}
We begin by providing details on how crimes are classified as occurring in either a public or residential space in both the administrative data from NYC, Los Angeles and Chicago as well as the NCVS.

\subsubsection{Administrative Crime Data}
Crimes known to law enforcement were gathered using incident-level administrative crime data released by the NYPD, the Los Angeles Police Department and the Chicago Police Department. Each city provides details about the location of a criminal incident, although the categories differ across cities. 

We use these data to determine whether a crime occurred inside a residence or in a public location.  In NYC, an incident is classified as residential if it occurred inside a private home, an apartment building or a public housing residence. In Los Angeles, we count as residential any crime in a residence, condomium/townhouse, foster home, dormitory, group home, shelter, mobile home, nursing home, transitional home, or yard. All other crimes are considered to have occurred in a public location. In Chicago, we count as residential any crime in a residence, apartment, basement, Chicago Housing Authority structure, college residence, hotel, residential driveway, or nursing home. All other crimes are considered to have occurred in a public location. 

\subsubsection{National Crime Victimization Survey}
Both the 2015-2019 and 2020 waves of the NCVS ask respondents to identify the type of location in which they were victimized. Using the 2015-2019 NCVS, we consider a crime to have occurred in a residential setting if the crime occurred at our near the victim's home or at or near a friend, neighbor or relative's house. Crimes that occurred in a commercial place, a parking lot, a school or another unspecified location are assumed to have occurred in a public location. 

The 2020 NCVS elicits more granular information on crime locations than the 2015-2019 NCVS.  We map the more granular 2020 location categories on to the less granular 2019 location categories as follows. Crimes that occurred in these locations are considered to be residential crimes: 1) in one's own dwelling, own attached garage or enclosed porch, 2) in a detached building on one's own property, 3) in a vacation/second home, 4) in a hotel or motel room that the respondent was staying in, 5) in one's own yard or driveway, 6) in a hallway, laundry room or storage area in one's own apartment building or 7) in a yard belonging to one's own apartment building. All other crimes are assumed to have occurred in a public location.

\subsection{Safegraph data}
We use the SafeGraph Neighborhood Patterns data \citep{sg_neighb} to calculate a count of people in each Census Block Group (CBG) every month. We build this off of the ``device home areas'' column, which gives a count of visitors by the visitors' home CBG. The home CBG is needed for weighting: we scale the number of visits from each CBG by the inverse of the sampling probability in that CBG, following \citet{sg_scale}. CBG populations are downloaded from SafeGraph's Open Census data, and the counts of devices in each CBG comes from SafeGraph's monthly Neighborhood Home Panel Summary files. The population of each CBG from Census, along with SafeGraph's reported device sample within each origin CBG, allows us to calculate the sampling probability each month.


\clearpage
\newpage 
%%%%%%%%%%%%%%%%%%%%%%%%%%%%%%%%%%%%%%%%%%%%%%%%%%%%%%%%%%%%%%
%%%%%%%%%%%%%%%%%%%%%%%%%%%%%%%%%%%%%%%%%%%%%%%%%%%%%%%%%%%%%%
%%%%%%%%%%%%%%%%%%%%%%%%%%%%%%%%%%%%%%%%%%%%%%%%%%%%%%%%%%%%%%
%%%%%%%%%%%%%%%%%%%%%%%%%%%%%%%%%%%%%%%%%%%%%%%%%%%%%%%%%%%%%%
\section{City-Specific Results} \label{sec:appendix2}
%%%%%%%%%%%%%%%%%%%%%%%%%%%%%%%%%%%%%%%%%%%%%%%%%%%%%%%%%%%%%%
%%%%%%%%%%%%%%%%%%%%%%%%%%%%%%%%%%%%%%%%%%%%%%%%%%%%%%%%%%%%%%
%%%%%%%%%%%%%%%%%%%%%%%%%%%%%%%%%%%%%%%%%%%%%%%%%%%%%%%%%%%%%%
%%%%%%%%%%%%%%%%%%%%%%%%%%%%%%%%%%%%%%%%%%%%%%%%%%%%%%%%%%%%%%
For brevity, in the main body of the paper, we present aggregated results for NYC, Los Angeles and Chicago as a whole. In this section, we present each set of results --- changes in mobility, reported crimes and street crime victimization risk --- for each of the three cities individually. While there is some variation in the evolution of crime rates and victimization risk among the three cities, the findings are substantively similar and suggest that some of the dynamics set in motion by the pandemic have been fairly universal.

In all three cities, mobility dropped considerably in March and April 2020. While foot traffic began to recover during the summer, each of the three cities ended the year with notably lower foot traffic than during the same months in 2019. Along with mobility, violent street crimes fell dramatically in April 2020 relative to April 2019 --- by 50\%, 20\% and 43\% --- in NYC, Los Angeles and Chicago, respectively. In all three cities, street crimes began to converge back to pre-pandemic levels over the summer.  However, by years' end, street violence remained approximately 10-15\% lower than it had been in 2019. Overall, summing across the March through December period,  public violence per 1,000 residents declined by 25\%, 7\% and 24\% in NYC, Los Angeles and Chicago, respectively.

With respect to public violence, all three cities experienced an upward shift in risk of victimization in 2020. However, the initial increase was the largest in Los Angeles where street crimes fell less sharply than in NYC or Chicago despite a sizable decline in outdoor activity. In Chicago, the change in risk was quite modest with risk increasing year-over-year by less than 5\% in most months. Despite the fact that each of the three cities faced a qualitatively different shift in victimization risk, in all three cities the risk of victimization and the crime rate diverged markedly in 2020. Overall, summing across the March through December period, the risk of public violence increased by 10\%, 23\% and 3\% in NYC, Los Angeles and Chicago, respectively.

\begin{figure}[h!]
   \caption{Violent crime and foot traffic in New York City}
     \hspace*{-1cm}
     \begin{tabular}{cc}
       \includegraphics[scale=0.6]{figs/nyc_effective_visitors1m.pdf}  & \includegraphics[scale=0.6]{figs/nyc_violentXoutside.pdf} \\ 
      (A) SafeGraph foot traffic &
       (B) Violent crime  \\
           \multicolumn{2}{c}{\includegraphics[scale=0.6]{figs/nyc_risk.pdf}} \\ 
     \multicolumn{2}{c}{(C) Risk} \\ 
    \end{tabular}
    \label{fig:chi_safegraph}
    \vspace*{-4mm}  \\ 
        \newline 
    Note: Figure plots the monthly change in foot traffic using Safegraph data (Panel A), the monthly number of non-residential (public) violent crimes (Panel B) and the monthly number of non-residential (public) violent crimes per 100,000 visitors based on mobility data from Safegraph (Panel C). Data are presented for NYC. In each plot, data are presented separately for 2019 (the blue line) and 2020 (the red line). 

\end{figure}

\clearpage
\begin{figure}[h!]
   \caption{Violent crime and foot traffic in Los Angeles}
     \hspace*{-1cm}
     \begin{tabular}{cc}
       \includegraphics[scale=0.6]{figs/la_effective_visitors1m.pdf}  & \includegraphics[scale=0.6]{figs/la_violentXoutside.pdf} \\ 
      (A) SafeGraph foot traffic &
       (B) Violent crime  \\
           \multicolumn{2}{c}{\includegraphics[scale=0.6]{figs/la_risk.pdf}} \\ 
     \multicolumn{2}{c}{(C) Risk} \\ 
    \end{tabular}
    \label{fig:chi_safegraph}
    \vspace*{-4mm}  \\  
        \newline 
    Note: Figure plots the monthly change in foot traffic using Safegraph data (Panel A), the monthly number of non-residential (public) violent crimes (Panel B) and the monthly number of non-residential (public) violent crimes per 100,000 visitors based on mobility data from Safegraph (Panel C). Data are presented for Los Angeles. In each plot, data are presented separately for 2019 (the blue line) and 2020 (the red line). 
\end{figure}

\clearpage
\begin{figure}[h!]
   \caption{Violent crime and foot traffic in Chicago}
     \hspace*{-1cm}
     \begin{tabular}{cc}
       \includegraphics[scale=0.6]{figs/chi_effective_visitors1m.pdf}  & \includegraphics[scale=0.6]{figs/chi_violentXoutside.pdf} \\ 
      (A) SafeGraph foot traffic &
       (B) Violent crime  \\
           \multicolumn{2}{c}{\includegraphics[scale=0.6]{figs/chi_risk.pdf}} \\ 
     \multicolumn{2}{c}{(C) Risk} \\ 
    \end{tabular}
    \label{fig:chi_safegraph}
    \vspace*{-4mm}  \\ 
        \newline 
    Note: Figure plots the monthly change in foot traffic using Safegraph data (Panel A), the monthly number of non-residential (public) violent crimes (Panel B) and the monthly number of non-residential (public) violent crimes per 100,000 visitors based on mobility data from Safegraph (Panel C). Data are presented for Chicago. In each plot, data are presented separately for 2019 (the blue line) and 2020 (the red line). 
 
\end{figure}



\newpage
\clearpage
%%%%%%%%%%%%%%%%%%%%%%%%%%%%%%%%%%%%%%%%%%%%%%%%%%%%%%%%%%%%%%
%%%%%%%%%%%%%%%%%%%%%%%%%%%%%%%%%%%%%%%%%%%%%%%%%%%%%%%%%%%%%%
%%%%%%%%%%%%%%%%%%%%%%%%%%%%%%%%%%%%%%%%%%%%%%%%%%%%%%%%%%%%%%
%%%%%%%%%%%%%%%%%%%%%%%%%%%%%%%%%%%%%%%%%%%%%%%%%%%%%%%%%%%%%%
\section{Robustness} \label{sec:appendix1}
%%%%%%%%%%%%%%%%%%%%%%%%%%%%%%%%%%%%%%%%%%%%%%%%%%%%%%%%%%%%%%
%%%%%%%%%%%%%%%%%%%%%%%%%%%%%%%%%%%%%%%%%%%%%%%%%%%%%%%%%%%%%%
%%%%%%%%%%%%%%%%%%%%%%%%%%%%%%%%%%%%%%%%%%%%%%%%%%%%%%%%%%%%%%
%%%%%%%%%%%%%%%%%%%%%%%%%%%%%%%%%%%%%%%%%%%%%%%%%%%%%%%%%%%%%%
In this section we provide alternative results using different methods to calculate victimization risk. First, we use survey data from the ATUS rather than Safegraph mobility data to activity-adjust the number of crimes. Next, we use a narrower definition of public spaces, excluding crimes committed indoors in public locations.


\subsection{Alternative Measure of Victimization Risk}
Our primary estimate of the risk of public violence is computed using Safegraph data which is the only source of mobility data which is publicly available prior to 2020. In this appendix, we re-compute our measure of victimization risk drawing on survey data from the American Time Use Survey (ATUS). To protect respondent anonymity, the survey data do not contain sufficient geographic detail to identify respondents living in NYC, Los Angeles and Chicago. We therefore focus on respondents who were living in metropolitan areas in 2019 and 2020. \autoref{fig:atus_risk} is identical to Panel B of \autoref{fig:city_crime_combined} except that ATUS data is substituted for Safegraph data in computing victimization risk. The $y$-axis represents the risk per 1 million hours spent outdoors.

  \begin{figure}
    \begin{center}
    \includegraphics[scale=1]{figs/atus_risk_combined.pdf}
    \caption{Risk in New York, Los Angeles, and Chicago using ATUS}
    \label{fig:atus_risk}
    \end{center}
            \vspace*{0mm}  \\ 
        \newline 
Note: Figure plots the number of non-residential (public) violent crimes per 1 million person-hours outside based on survey data from the American Time Use Survey (ATUS) and crime for the combined study sample (NYC $+$ Los Angeles $+$ Chicago). Data are presented separately for 2019 (the blue line) and 2020 (the red line). The ATUS was not administered in April 2020.
\end{figure}



As the ATUS was temporarily suspended from mid-March until mid-May 2020 due to public health concerns, we are not able to observe the change in victimization risk that occurred just after the beginning of the pandemic.\footnote{See: \url{https://www.bls.gov/tus/covid19.htm}.} However, for the remainder of the year, estimates of victimization risk using the Safegraph data and the ATUS survey data are substantively the same: that the risk of street crime victimization increased markedly in Spring 2020 and remained elevated --- by approximately 10\% --- during the remainder of the year.

 
\subsection{Alternative Measure of Public Locations}

We begin by plotting the change in the incidence and risk of public violence using an alternative definition of what it means for the crime to have occurred in public.  \autoref{fig:city_crime_alt_combined} is analogous to \autoref{fig:city_crime_combined} except it uses a more limited definition of public violence. In this figure, we count only crimes that occurred on streets, alleyways, public parks, or transit system. Estimated changes in risk are extraordinarily similar to the main estimates reported in the paper.

%\newpage 
%\clearpage
\begin{figure}[h!]
    \caption{Change in Public Violence and Victimization Risk using Alternative Definition of Public Crimes --- New York, Los Angeles and Chicago (2019-2020)}
     \begin{center}
     \begin{tabular}{c}
     \includegraphics[scale=0.6]{figs/crimes_alt_combined.pdf} \\ 
     (A) Violent crimes \\ 
     \includegraphics[scale=0.6]{figs/risk_alt_combined.pdf} \\ 
     (B) Risk, SafeGraph \\ 
    \end{tabular}
         \end{center}
    \label{fig:city_crime_alt_combined}
    \vspace*{0.5cm}  \\ 
        \newline 
    Note: This plot is analogous to \autoref{fig:city_crime_combined} except it uses a more limited definition of public violence. In each city, we count only crimes that occurred on streets, alleyways, public parks, or transit system. 
\end{figure}

\subsection{Changes in Outdoor Time Use}
Here, we address the possibility that the types of activities people engaged in while in public may have changed in 2020. To the extent that people spent more time engaged in activities that carry a higher degree of inherent risk, the increase in victimization risk that we observe could be an artifact of a shift in time use. In Appendix Figure \ref{fig:atus_where}, we plot the share of time spent in retail establishments and restaurants, at work or school, in one's car and walking for the 2015-2020 period. While the overall amount of time spent outdoors in 2020 was lower than in previous years, there is no clear change in the mix of activities. Of special note is outdoor time spent walking which has hovered between 16\% and 19\% throughout the 2015-2020 period.

\begin{figure}
\begin{center}
\includegraphics[scale=1]{figs/atus_where_change.pdf}
\caption{Change in outdoor time use}
\label{fig:atus_where}
\end{center}
        \vspace*{0mm}  \\ 
    \newline 
Note: This shows changes in where ATUS respondents spent their time away from home, using broad categories of their responses to the ``where'' question.
\end{figure}


\newpage
\clearpage
%%%%%%%%%%%%%%%%%%%%%%%%%%%%%%%%%%%%%%%%%%%%%%%%%%%%%%%%%%%%%%
%%%%%%%%%%%%%%%%%%%%%%%%%%%%%%%%%%%%%%%%%%%%%%%%%%%%%%%%%%%%%%
%%%%%%%%%%%%%%%%%%%%%%%%%%%%%%%%%%%%%%%%%%%%%%%%%%%%%%%%%%%%%%
%%%%%%%%%%%%%%%%%%%%%%%%%%%%%%%%%%%%%%%%%%%%%%%%%%%%%%%%%%%%%%
\section{Alternative Explanations} \label{sec:appendix_alternative}
%%%%%%%%%%%%%%%%%%%%%%%%%%%%%%%%%%%%%%%%%%%%%%%%%%%%%%%%%%%%%%
%%%%%%%%%%%%%%%%%%%%%%%%%%%%%%%%%%%%%%%%%%%%%%%%%%%%%%%%%%%%%%
%%%%%%%%%%%%%%%%%%%%%%%%%%%%%%%%%%%%%%%%%%%%%%%%%%%%%%%%%%%%%%
%%%%%%%%%%%%%%%%%%%%%%%%%%%%%%%%%%%%%%%%%%%%%%%%%%%%%%%%%%%%%%
In the main body of the paper, we observed that while the official crime rate was lower in 2020 than it had been in 2019, the risk of public violence increased. In this appendix, we provide evidence that this increase in risk is unlikely to be an artifact of changes in crime reporting behavior by victims or compositional shifts in the pool of people spending time outside.

\subsection{Victim Reporting} \label{sec:reporting}
To the extent that victims became more reluctant to report crimes to law enforcement due to public health risks, a legitimacy crisis in policing \citep{tyler2004enhancing, tankebe2014police, wolfe2016effect} sparked by the killing of George Floyd \citep{nix2021more}, or the perception that police and prosecutors were otherwise occupied, declining crime might be a mechanical artifact of a change in reporting behavior \citep{van1979victim, levitt1998relationship, davis2003willingness}. In using the official data and transforming it to account for changes in mobility, we might then understate the increase in victimization risk. Similarly, we could overstate the risk of crime if reporting increased.
\begin{figure}[h!]
     \begin{center}
    \caption{Share of Outdoor Violent Crimes Reported to Police --- 2015-2019 and 2020 National Crime Victimization Surveys}
    \includegraphics[scale=0.7]{figs/reporting_rate.pdf}
    \label{fig:reporting_rate}
     \end{center}
        \vspace*{-8mm}  
        \newline 
Note: Figure plots the percentage of non-residential (public) violent crimes reported to law enforcement in the 2015-2019 and 2020 waves of the National Crime Victimization Survey (NCVS). For each survey wave, the point estimate is provided along with a 95\% confidence interval.
\end{figure}
To address this concern, we use an NCVS question on whether victims reported crimes to law enforcement. Appendix Figure \ref{fig:reporting_rate} presents the share of outdoor violent crimes that victims reported crimes to police among respondents interviewed in the 2015-2019 and 2020 waves of the survey. To maintain consistency with our administrative data from large cities, we focus on survey respondents living in metropolitan areas. Contrary to concerns that crime reporting may have fallen during the pandemic, the rate at which outdoor violent crimes were reported to the police was not unusual in 2020.


\subsection{Selection} \label{sec:selection}
A second potential issue is selection, the possibility that the composition of individuals who spent time outdoors and were therefore at risk of street crime victimization changed in 2020. Since the pandemic has had unequal public health impacts according to gender, race, and especially age \citep[e.g.][]{hutchins2020covid, miller2021estimated}, each of which is among the strongest predictors of victimization risk \citep[e.g.][]{perkins1997age}, this is a critical concern. In the extreme, it is possible that the aggregate risk of victimization could increase while decreasing for every demographic group, a case of Simpson's paradox \citep{blyth1972simpson}. In this section, we motivate three tests for the importance of selection using national data on time use from the ATUS, city-specific data on realized crime victimization, and within-neighborhood changes in victimization risk derived from both municipal microdata and Safegraph patterns data. None of the three tests suggests that the increase in victimization risk that we observe after March 2020 is likely to be an artifact of selection.

\subsubsection{Test of demographic selection using national survey data} \label{sec:demog_selection}
Our first test of selection uses a DFL-style reweighting exercise \citep{dinardo1996labor} to assess the contribution of selection into activity to the crime rate. This computation addresses how crime would have changed if activity had evolved as observed and victimization rates were fixed at their pre-2020 levels. 

Using data from the 2015-2019 NCVS we compute public victimization rates for each of eighteen age-race-gender cells defined by the intersection of race (White, Black, Other), gender (male, female) and age ($<$ 25, 26-49 and $>$ 50).%\footnote{We use less granular age categories to guard against drawing inferences from small bins in the ATUS data.}$^{,}$
%\footnote{In Appendix Figure \ref{fig:victimization_by_inc}, recognizing the importance of income, we replicate this analysis using age-race-income groups instead. Results are substantively similar.}  
These data are combined with estimates from the ATUS, which gives the share of time spent outside in 2019 and 2020 for each group. In \autoref{fig:victimization_by_group2} we plot baseline time use  (Panel A), the change in time use between 2019 and 2020 (Panel B) and baseline victimization rates (Panel C) for each of the eighteen demographic groups. 

Prior to the pandemic, though older people spent less time outdoors than younger people, there was little variation in time spent outdoors among the eighteen groups. How did time spent outside change after the COVID-19 pandemic? The decline in the share of time spent in public spaces was fairly universal and, across all groups, there was a 16\% decrease in waking hours spent outdoors in 2020. However, several dimensions of heterogeneity are notable. First, the behavioral response to the pandemic was notably muted among prime-age Black men. Second, the largest disruption to pre-pandemic behavior occurred in individuals under the age of 25, some of whose routine activities were disrupted by school closures. 

In Panel C, we consider hetereogeneity in baseline victimization risk, measured during the pre-pandemic period. In the figure, the blue bars represent the number of violent victimizations occurring in public locations per 1,000 people in each group. The red bars use the data from Panel A to compute activity-adjusted victimization risk. This computation shrinks or inflates each group’s victimization rate according its outdoor activity relative to the population average. For example, white males under 25 spend more time outside: 10.9 hours compared to an average of 9.6 hours in the ATUS population. Their activity-adjusted victimization rate is scaled by $9.6/10.9$ or about 0.88. Accounting for their increased activity makes their activity-adjusted victimization rate lower.

As is evident from the raw victimization rates, there is considerable heterogeneity in victimization rates across the eighteen groups with an especially strong age gradient. While young men face annual outdoor violent victimization rates of nearly 30 per 1,000, individuals over the age of 50 face rates that are uniformly below 10 per 1,000 and are sometimes as low as 5 per 1,000. For individuals below age 50, men face uniformly higher victimization rates than women with the gender gradient attenuating somewhat in the highest age category. When we adjust victimization rates for differences in time spent outside, the differences between groups narrow albeit only slightly. An implication of this result is that, prior to the pandemic, between-group differences in victimization risk are not well explained by between-group differences in mobility. 

Given the outsize importance of age in predicting victimization risk and the fact that time spent outside declined more among younger individuals than among older individuals, selection effects point to a \emph{reduction} in baseline risk in 2020. To formalize this observation, we predict the change in victimization risk that would arise from an observed shift in the demographics of individuals spending time outside their homes. We compute three quantities: (i) the total number of public violent crimes experienced by Americans in the pre-pandemic period, (ii) the expected change in victimization given the observed change in time spent in public for each demographic subgroup and (iii) the expected change in victimization assuming that every demographic subgroup reduced their time spent in public by the same amount in 2020. The difference between (ii) and (iii) indicates the importance of demographic selection effects. 

Below we motivate a formal model to account for the impact of compositional changes on victimization risk. We begin by noting that the annual number of outdoor crimes experienced by Americans in 2019 can be expressed as:
\begin{equation} \label{eq:selection1}
C_{2019} = \sum_{j \in J} n_{j,2019} \times ShareOutside_{j,2019} \times v_{j,2019}
\end{equation}
where $n_{j,2019}$ is the number of people in group $j$ in 2019, $ShareOutside_{j,2019}$ is the share of time individuals in group $j$ spent outside in 2019 and $v_{j,2019}$ is the group's outdoor victimization rate, defined as the number of outdoor crimes divided by the number of person-hours spent at risk:
\[
v_j = \frac{OutdoorCrimes_{j,2019}}{n_{j,2019}  \times ShareOutside_{j,2019}}
\]

Next we employ this notation to compute the two counterfactual quantities that we use to identify the importance of compositional effects. Let $\Delta_j$ be the change in the proportion of outdoor time, where $\Delta_j= ShareOutside_{j,2020} - ShareOutside_{j,2019}$.\footnote{In practice, to reduce noise, we use the 2015-2019 pre-period.} The number of crimes that would have been experienced by Americans, holding victimization rates constant but allowing behavioral responses to the pandemic (i.e., the change in the amount of time spent in public spaces) to vary by group is given by $\hat{C}_{2020}$:
\begin{equation} \label{eq:selection2}
\hat{C}_{2020} = \sum n_{j,2020} \times (ShareOutside_{j,2019} +\Delta_j) \times v_{j,2019}
\end{equation}
Equations \href{eq:selection1}{(\ref{eq:selection1})} and \href{eq:selection2}{(\ref{eq:selection2})} are identical except that, in \href{eq:selection2}{(\ref{eq:selection2})}, each group changes the share of time spent in public spaces in 2020 according to $\Delta_{j}$. $\hat{C}_{2020}$ is the predicted number of outdoor violent crimes in metropolitan areas of the United States, holding hourly victimization risk constant.

Next, we consider the counterfactual victimization rate under the constraint that all groups have an identical response to the pandemic (i.e., all groups reduce their time spent outdoors by the same quantity), substituting $\bar{\Delta}$, the population-weighted average across all $J$ groups, for $\Delta_j$:
\begin{equation} \label{eq:selection3}
C^*_{2020} = \sum n_{j,2020} \times (ShareOutside_{j,2019} + \bar{\Delta}) \times v_{j,2019}
\end{equation}
The number of outdoor violent crimes under this ``alternative pandemic", $C^*_{2020}$, is generated by constraining each group to have an identical response with respect to time use. The ratio of $\hat{C}_{2020}$ to $C_{2020}^{*}$ provides a measure of the importance of compositional effects---the additional number of crimes per day that we would expect, given constant victimization risk, from a heterogeneous as compared to a homogenous behavioral response to the pandemic. 

%We use the NCVS to obtain an estimate of $v_{i}$ for each of eighteen age-gender race groups (\autoref{fig:victimization_by_group2}) and the ATUS to estimate $\bar{\Delta}$ as well as $\Delta_{i}$ for each demographic subgroup. We begin by computing the total number of crimes experienced by Americans in the pre-pandemic period, $C_{2019}$. Taking a weighted sum over each of the groups, we obtain an annual estimate of 1.34 million outdoor violent crimes.\footnote{Formally $v_{i}$ is obtained by dividing the victimization rate for each group by the number of hours spent outside by individuals in each group during the six-month recall period in the NCVS.} Next, we compute $ \hat{C}_{2020}$ and $C_{2020}^{*}$. Using observed time use in 2020 and pre-determined victimization risk, we estimate that $\hat{C_{2020}}$ = 923,000 outdoor violent crimes. This number is 31\% smaller than the number of crimes measured in 2019 which is consistent with time spent outdoors having declined. Finally, constraining each group to have changed their public time use by $\bar{\Delta}$ = 15.8\%, we obtain an estimate of $C_{2020}^{*}$ = 970,000 outdoor violent crimes. The difference between $C_{2020}^{*}$ and $\hat{C}_{2020}$ is approximately 47,000 crimes, indicating that selection effects, captured using changes in the demography of individuals spending time outdoors, would predict a 5\% \textit{decrease} in crime. While the pandemic led to some re-sorting of individuals in outdoor spaces in 2020, on net, compositional changes, based on key observable dimensions of victimization risk, do not predict a large change in offending.



In order to generate estimates according to this framework, we begin by computing the total number of crimes experienced by Americans living in large metro areas in the pre-pandemic period. Summing over each of the eighteen groups, we obtain an annual estimate of 1.34 million outdoor violent crimes. Next, holding victimization risk fixed, we estimate that 923,000 outdoor violent crimes would have accrued on the basis of the observed reduction in time spent in public across the eighteen demographic groups. This number is 31\% less than the number of crimes in 2019, reflecting the reduction in time spent in public. Finally, constraining each group to have changed their public time use by 15.8\%, the mean decline in time spent outside across all groups, we obtain an estimate of 970,000 outdoor violent crimes. The difference between the counterfactual condition in (3) and counterfactual condition in (4) is approximately 47,000 crimes, indicating that demographic selection effects would predict a 5\% \textit{decrease} in crime. This is consistent with our casual observation that, if anything, the risk set grew relatively \emph{older} in 2020. %Notably, when we make the same adjustment using age-race-income groups, selection effects are effectively zero --- see Appendix Figure \ref{fig:victimization_by_inc}. 


\begin{figure}
     \begin{center}
    \caption{Victimizations per 1,000 Population for Selected Demographic Groups, 2015-2019 National Crime Victimization Survey}
    \begin{tabular}{c}
    \hspace*{-3cm}
    % \includegraphics[scale=.55]{figs/VictimizationATUSNCVSfig.pdf}
    \includegraphics[scale=.55]{figs/VictimizationATUSNCVSfig_ver2.pdf}
    \end{tabular}
    \label{fig:victimization_by_group2}
         \end{center}
        \vspace*{0mm}  \\ 
        \newline 
Note: Figure provides descriptive evidence on time use and victimization rates prior to the COVID-19 pandemic. Panel (A) summarizes the percentage of time spent away from one's home or yard using the 2015-2019 American Time Use Survey (ATUS) for each of eighteen age-gender-race groups.  Panel (B) summarizes non-residential (public) violent victimizations per 1,000 population using the 2015-2019 waves of the National Crime Victimization Survey (NCVS). The blue bars are the unadjusted rates. The purple bars are rates of public violent victimizations per person-hour spent in public.
\end{figure}


\newpage 
\subsubsection{Test of income-based selection using national survey data}
While age, race and gender are particularly salient predictors of victimization, socioeconomic status is also a noteworthy predictor. Given that higher income earners were more likely to be able to work from home during the COVID-19 pandemic, it is possible that the mix of people spending time outdoors in 2020 shifted towards lower-income workers. 

In order to test for whether any such shifts occurred and whether such a shift might compromise our estimates, we repeat the analysis presented in Section \ref{sec:demog_selection} for age-race-income groups where we divide the population into those earning above \$50,000 and those earning below \$50,000. In Panel A, we plot the number of hours spent outside by each of the 12 demographic groups. In Panel B, we plot the change in this share between 2019 and 2020. In Panel C, we compute activity-adjusted victimization rates for each group. 

Finally, as in our main analysis, we use this information to compute a DFL-style re-weighting estimate. The difference between $\hat{C}_{2020}$ and $C_{2020}^{*}$ implies that selection effects with these categories would predict a 0.1\% decrease in crime. Overall, while there was some re-sorting of time spent outside during the COVID-19 pandemic, this re-sorting does not predict higher crime victimization.  

\begin{figure}
     \begin{center}
    \caption{Victimizations per 1,000 Population for Selected Demographic Groups, 2015-2019 National Crime Victimization Survey}
    \begin{tabular}{c}
    \hspace*{-3cm}
    \includegraphics[scale=.55]{figs/VictimizationATUSNCVSfig_incomesplit.pdf}
    \end{tabular}
    \label{fig:victimization_by_inc}
         \end{center}
        \vspace*{0mm}  \\ 
        \newline 
Note: Figure provides descriptive evidence on time use and victimization rates prior to the COVID-19 pandemic, split by income. Panel (A) summarizes the percentage of time spent away from one's home or yard using the 2015-2019 American Time Use Survey (ATUS) for ge-income-race groups.  Panel (B) summarizes non-residential (public) violent victimizations per 1,000 population using the 2015-2019 waves of the National Crime Victimization Survey (NCVS). The blue bars are the unadjusted rates. The maroon bars adjust for hours spent in public.
\end{figure}

\newpage 
\subsubsection{Direct test based on victim demographics in NYC and Los Angeles} \label{selection2}

%Finally, we assess the presence of demographic selection effects directly using crime microdata from NYC and Los Angeles, where data on the demography of crime victims are available. If our finding of an increase in risk is driven by selection, crime victims in 2020 should, in general, be those who have higher baseline victimization risk than in the past. Evidence from this analysis is presented in Appendix \ref{selection2} and suggests that changes in the demography of potential crime victims is unlikely to account for a meaningful share of the large increase in victimization risk that we observe in the data. Consistent with the above analysis, 2020 crime victims had lower, not higher, historical rates of victimization.

We can also assess the size of potential selection effects directly using crime microdata from NYC and Los Angeles. If our finding of an increase in risk is driven by selection, crime victims in 2020 should, in general, be at a higher risk of victimization than they were in the past. We can probe this possibility using data on victim demographics. As in the main results, we restrict to violent crimes that occurred outdoors.

We match each victim to their expected victimization rate according to their age, race, and gender based on data from the 2015-2019 National Crime Victimization Surveys (NCVS), restricting to outdoor crimes and to respondents living in cities with populations over one million. We use 24 demographic cells in total, representing the interaction of three race/ethnicity groups (White, Black, Hispanic), two genders (male, female), and four age groups (12-18, 18-24, 25-44, 45+).\footnote{The last age group is an imperfect match with the NCVS, which bins all ages 35-49. Victims 45+ in the NYC and LA crime data are matched with NCVS respondents over 50.} More formally, the NCVS data yields $v_g$, a victimization rate for group $g$, with $g \in \{ 1,...,24\}$ where, e.g., $v_1$ could be the victimization rate White females under 18 which is 20 per 1,000 (per year). Next, we assign the victimization rate $v_g$ to each crime in the NYC and LA data according to victim demographics and calculate the average victimization rate for each quarter $q$:
\begin{equation}
\overline{v}_q = \frac{ \sum_{g=1}^{24} v_g * N_{g,q}}{N_q}
\end{equation}
where  $N_{g,q}$ is the total number of victims who belong to group $g$ in quarter $q$ and $N_q$ is the total count of crimes in quarter $q$ where we observe victim demographics. The time series of $\overline{v}_q$, the average expected victimization rate, shows us how selection into crime victimization based on risk changes over time.

Using data for the combined two-city sample, we plot the average expected victimization rate of victims over time in \autoref{fig:historical_avg_vic_rates}. The series drops sharply in the second quarter of 2020 and remains lower, decreasing from 15.1 to 14.7 violent offenses per 1,000 people across years. This suggests that, if anything, crime victims of 2020 were \emph{less likely} to have been victimized than in previous years. This implies that within-group increases in risk were likely larger than in our aggregated estimates. Taken together, the available evidence suggests that compositional changes in the demography of potential crime victims is unlikely to account for a meaningful share of the large increase in victimization risk that we observe in the data.

\begin{figure}[h!]
    \caption{Average historical victimization rates}
    \label{fig:historical_avg_vic_rates}
    \begin{center}
    \includegraphics{figs/historical_avg_vic_rates.pdf}
    \end{center}
\end{figure}


%In this appendix we provide additional results on the demographic composition of crime victims in NYC and Los Angeles prior to and during the COVID-19 pandemic. We also provide descriptive evidence on the time-path of disturbances to public health as the timing of the impact of the COVID-19 pandemic differed among the three cities in our sample. 

%\autoref{fig:victim_demog} shows additional details on victim demographics, drawing from microdata in NYC and Los Angeles which include information on the age, race and gender of crime victims. Across the two cities, changes in the age distribution of victims in 2020 were  modest. There was a small decrease in the relative shares of victims under the age of 18 or between the ages of 18 and 24. As these are the age groups that are most likely to be victimized at baseline, if anything, it appears as though changes in the demography of victims point to reduced risk of victimization rather than an increase in victimization risk.

\begin{comment}
\begin{figure}[h!]
\begin{center}
    \caption{Victim demographics, New York and Los Angeles}     \label{fig:victim_demog}
    %  \hspace*{-.6cm}
     \begin{tabular}{cc}
       \includegraphics[scale=0.55]{figs/vic_ages_male_minus.pdf}  & \includegraphics[scale=0.55]{figs/vic_ages_female_minus.pdf} \\ 
       (A) Age, male victims &
       (B) Age, female victims   \\
       \includegraphics[scale=0.55]{figs/vic_races_male_minus.pdf}  & \includegraphics[scale=0.55]{figs/vic_races_female_minus.pdf} \\ 
       (C) Race, male victims &
       (D) Race, female victims    \\
    \end{tabular}
    \end{center}
    \vspace*{0.5cm}  \\ 
    \small
    Note: These plots show how the composition of violent crime victims changed in New York City and Los Angeles, the two cities where victim demographics are available. 
\end{figure}


\begin{figure}
\begin{center}
    \caption{Historical victimization rates of victims, New York City and Los Angeles}
    \includegraphics[scale=0.7]{figs/historical_avg_vic_rates.pdf}
    \label{fig:historical_avg_vic_rates}
\end{center}
        \vspace*{-4mm}  \\ 
        \newline 
    Note: Figure presents average historical victimization rates of crime victims from the NYPD and LAPD incident data. The historical victimization rates are calculated using data from the 2015-2019 waves of the NCVS and victim demographics are reported in the incident data. Let $g$ index demographic groups and $v_g$ be the victimization rate for group $g$. The average historical victimization rate in quarter $q$ is given by:
\[
\overline{v}_q = \frac{ \sum_{g} v_g * N_{g,q}}{N_q}
\]
where $N_{g,q}$ is crimes against group $g$ in quarter $q$ and $N_q$ is total crimes.
\end{figure}
\end{comment}

\subsubsection{Test of geographic selection}
Given wide variation in crime rates across different communities, another potential source of selection is geographic. If time spent outdoors changed more in some communities than in others---for example, if individuals living in higher poverty communities were less able to shelter at home---changes in risk could be an artifact of between-community differences in victimization risk. We assess whether the estimates presented in Figure 2 attenuate when we focus on the change in victimization \emph{within a given community}. This analysis is especially salient as Census block group fixed effects explain nearly 80\% of the variation in crime risk among our three cities.

We study within-community changes in the following specification:
\begin{equation}
Risk_{cmy} = \sum_{k \in \{1,...,12\}} \beta_k 1\{m=k, y=2020\} + \alpha_{cm} + \epsilon_{cmy} 
\end{equation}
where $Risk_{cmy}$ gives risk per 100,000 visitors in census block group $c$ during month $m$ in year $y$. $\alpha_{cm}$ denote Census block group by month fixed effects, so that the coefficients $\beta_k$ track the within-area change in risk compared to the same month in 2019. Observations are weighted by the number of visitors, and standard errors are clustered at the Census block group level. 

\begin{figure}
     \begin{center}
    \caption{Victimization Risk, Conditional on Census Block Group Fixed Effects}
    \begin{tabular}{c}
     \includegraphics[scale=1]{figs/cbg_within_coefs.pdf}    \\
 
    \end{tabular}
    \label{fig:awesome}
         \end{center}
        \vspace*{0mm}  \\ 
        \newline 
Note: Figure plots coefficients on month-of-year dummy variables  from a regression of victimization risk on month, year and Census block group fixed effects. The horizontal line at $y$=0 indicates no change in victimization risk between 2019 and 2020, conditional on Census block group fixed effects. The 95\% confidence interval is plotted using standard errors clustered at the Census block group level.
\end{figure}


We plot coefficients $\beta_k$ and the associated 95\% confidence intervals in \autoref{fig:awesome}. The pattern in the coefficients is similar to those reported in \autoref{fig:city_crime_combined}, indicating that victimization risk rose most steeply in April and May 2020 but remained elevated throughout the remainder of the year. This suggests that the increase in risk that we observe is not driven by a shift in the geographic composition of the population at risk. 






%\input{app_figs}

\appendix 




\end{document}
