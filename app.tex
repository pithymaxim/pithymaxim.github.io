\newpage
\clearpage
\appendix



\singlespacing
%%%%%%%%%%%%%%%%%%%%%%%%%%%%%%%%%%%%%%%%%%%%%%%%%%%%%%%%%%%%%%
%%%%%%%%%%%%%%%%%%%%%%%%%%%%%%%%%%%%%%%%%%%%%%%%%%%%%%%%%%%%%%
%%%%%%%%%%%%%%%%%%%%%%%%%%%%%%%%%%%%%%%%%%%%%%%%%%%%%%%%%%%%%%
%%%%%%%%%%%%%%%%%%%%%%%%%%%%%%%%%%%%%%%%%%%%%%%%%%%%%%%%%%%%%%
\section{History of Measuring Activity-Adjusted Crime Risk} \label{sec:appendixA}
%%%%%%%%%%%%%%%%%%%%%%%%%%%%%%%%%%%%%%%%%%%%%%%%%%%%%%%%%%%%%%
%%%%%%%%%%%%%%%%%%%%%%%%%%%%%%%%%%%%%%%%%%%%%%%%%%%%%%%%%%%%%%
%%%%%%%%%%%%%%%%%%%%%%%%%%%%%%%%%%%%%%%%%%%%%%%%%%%%%%%%%%%%%%
%%%%%%%%%%%%%%%%%%%%%%%%%%%%%%%%%%%%%%%%%%%%%%%%%%%%%%%%%%%%%%
Researchers typically track changes in public safety using the crime rate: the number of crimes known to law enforcement divided by an area's population \citep{nolan2004establishing}. Crime rates are used to compare criminal activity across cities and to understand how public safety has changed over time. The advantage of the crime rate is that it is transparent and can be straightforwardly computed for individual cities as well as nationally using publicly-available data. 

The crime rate is a convenient albeit imperfect heuristic for public safety, a core concept but one that can be difficult to define and even harder to measure. While public safety can refer to a number of different ideas, a common conception employed in research and the policy world involves the risk of victimization for a typical citizen \citep{boggs1965urban, stipak1988alternatives, vaughan2020promise, ramos2021improving}. Victimization risk motivates the central statistics --- crime incidence and prevalence --- that are released by the U.S. Bureau of Justice Statistics in their annual distillation of the U.S. National Crime Victimization Survey, the sole national victimization survey in the United States. Victimization risk is likewise a key ingredient in how members of the public think about safety, especially when it comes to the risk of becoming the victim of a street crime \citep{ferraro1995fear, pickett2012reconsidering}.

Using the crime rate as a proxy for public safety has led to a litany of critiques in the criminology literature. The most common criticism of the crime rate is that crimes that become known to law enforcement --- the only city-level measure of victimization that is consistently available in the United States --- represent only a subset of criminal activity. In particular, researchers have worried about the ``dark figure of crime," the number of crimes which are unreported to and undetected by state or local police agencies \citep{biderman1967exploring, penney2014dark}. In the presence of victim underreporting and incomplete crime detection by police, the crime rate will underestimate the true risk of becoming a crime victim, a problem which is thought to be particularly large for stigmatized crimes like domestic violence and sexual assault \citep{felson2005reporting}, crimes with lower social costs and relatively low clearance rates like theft \citep{skogan1977dimensions} and crimes against juveniles who may be particularly apprehensive about interacting with the police \citep{finkelhor2001factors}.%
\footnote{In the United States, a related problem is that, despite national reporting standards set forth by the Federal Bureau of Investigation, inconsistent recording practices by local law enforcement agencies can distort the validity of between-city crime comparisons.} 
While the difficulty of accurately recording crime represents an important challenge to using the crime rate, this challenge is far from insurmountable. Using national survey data, it is possible to estimate the magnitude of the dark figure of crime, including hard-to-measure crimes like domestic violence \citep{bachman1994violence} or crimes against juveniles \citep{hashima1999violent}. Likewise, by focusing on crimes which tend to be consistently recorded --- for example, murder and motor vehicle theft --- measurement artifacts can be minimized.\footnote{Other challenges to using the crime rate to understand the risk of victimization include those created by population heterogeneity and differential selection into risky activities.}

In this paper, we focus on a deeper and more challenging issue in the measurement public safety --- a problem that has been noted by researchers for many years but which, due to severe data constraints remain unresolved. While measuring the crime rate's numerator has received the lion's share of scholarly attention, it is actually the fraction's denominator that poses the most salient challenge for researchers \citep{boggs1965urban}. That is, assuming that we have an accurate accounting of the number of crimes, against what reference group should that number be compared for measuring public safety? While population is a useful starting point, for several reasons it may be a poor proxy for the number of criminal opportunities. 

As has been recognized in previous research, a city's resident population is an imperfect proxy for the number of individuals who are at risk to become a crime victim \citep{boggs1965urban, boydell1969demographic, skogan1978victimization, stipak1988alternatives, newton2018macro,gerell2021does, vaughan2020promise, ramos2021improving}, particularly the victim of the types of street crimes that capture an outsize amount of fear in the public's imagination \citep{skogan1986fear, ferraro1987measurement}. For example, prior to the COVID-19 pandemic, Manhattan Island, home to approximately 1.7 million individuals, swelled to a population of approximately 3.4 million people during a typical workday. Clearly, deflating the number of crimes by 1.7 million leads to a crime rate that is biased upwards, an issue noted by \citeauthor{boggs1965urban} nearly sixty years ago.\footnote{As Boggs noted, ``spuriously high crime occurrence rates are computed for central business districts, which contain small numbers of residents but large numbers of such targets as merchandise on display, untended parked cars on lots, people on the streets, money in circulation, and the like.''}
At the same time, since people who work in Manhattan likely spend less time there, on average, than Manhattan residents, using 3.4 million would yield an underestimate. Ideally, we would deflate crime by the number of person-hours spent in Manhattan during a given period. However this figure is, for obvious reasons, difficult to estimate.

A related concern is that crime risk is, in large part, a function of its opportunity, an idea that criminologists generally refer to as routine activities theory \citep{cohen1979social, felson1987routine, branic2015routine}. This theory offers a parsimonious explanation for why juvenile offending peaks after school lets out \citep{fox1997after, fischer2018juvenile} and why vehicle thefts tend to be counter-cyclical, falling during recessions --- there are fewer vehicles to steal from downtown areas \citep{cook1985crime,bushway2012overall}. If a population changes its behavior due to an exogenous shock, the crime rate might change even if the risk of crime for those who are unresponsive to the shock remains constant. This issue will be especially pronounced in times where activity fluctuates; sharp changes in the crime rate could belie less radical changes in risk or vice versa. 

Prior literature has proposed several different innovations to better measure the risk of crime. Recognizing that the risk of exposure varies from crime to crime and might be poorly correlated with population, the earliest literature proposed a series of crime-specific denominators including vehicles registered for vehicle thefts \citep{lottier1938distribution, cohen1985risk}, female population for rape \citep{boggs1965urban} and the number of occupied housing units as a denominator for residential
burglary \citep{boggs1965urban, minnesota1977crime}. Recognizing the multi-dimensional nature of risk, others have proposed regression adjustment as a way to empirically model crime risk across population as well as other relevant denominators such as vehicles registered \citep{stipak1988alternatives}. 

Subsequent research has focused on time spent at and away from home as creating a salient measure of the risk of exposure. Starting with a seminal contribution by \citet{cohen1979social} which noted that individuals spend, on average, two-thirds of their time at home, one hour per day outdoors, and the remaining time indoors at a location that is not their own residence. Using these denominators, the authors calculated that, per unit of time, the risk of assault by a stranger on the street was more than 20 times greater than the risk of assault at home by someone who is known to the victim. While this result may seem perfectly intuitive today, at the time, this research was instrumental in overturning a common belief by scholars that home was the place where crime risk was greatest \citep{vaughan2020promise}. A recent innovation proposed by \citet{vaughan2020promise} and \citet{lemieux2012risk} is to use national survey data to measure changes in the time that individuals spend outside their homes. By comparing data on victimization from the National Crime Victimization Survey to data on time use from the ATUS, these papers have computed victimization rates for various activities, finding that victimization risk for the 2003-2008 period is approximately 8 per one million hours spent in public. We update and build upon this approach, augmenting national survey data from the ATUS with new data from mobile phone-based geo-location software.

\newpage 
%%%%%%%%%%%%%%%%%%%%%%%%%%%%%%%%%%%%%%%%%%%%%%%%%%%%%%%%%%%%%%
%%%%%%%%%%%%%%%%%%%%%%%%%%%%%%%%%%%%%%%%%%%%%%%%%%%%%%%%%%%%%%
%%%%%%%%%%%%%%%%%%%%%%%%%%%%%%%%%%%%%%%%%%%%%%%%%%%%%%%%%%%%%%
%%%%%%%%%%%%%%%%%%%%%%%%%%%%%%%%%%%%%%%%%%%%%%%%%%%%%%%%%%%%%%
\section{Data Details} \label{sec:safegraph_deets}
%%%%%%%%%%%%%%%%%%%%%%%%%%%%%%%%%%%%%%%%%%%%%%%%%%%%%%%%%%%%%%
%%%%%%%%%%%%%%%%%%%%%%%%%%%%%%%%%%%%%%%%%%%%%%%%%%%%%%%%%%%%%%
%%%%%%%%%%%%%%%%%%%%%%%%%%%%%%%%%%%%%%%%%%%%%%%%%%%%%%%%%%%%%%
%%%%%%%%%%%%%%%%%%%%%%%%%%%%%%%%%%%%%%%%%%%%%%%%%%%%%%%%%%%%%%
In this appendix we provide additional detail on the sources of data used in the paper and how our measures of crime and mobility were constructed from those data. 

\subsection{Defining Crime Locations}
We begin by providing details on how crimes are classified as occurring in either a public or residential space in both the administrative data from NYC, Los Angeles and Chicago as well as the NCVS.

\subsubsection{Administrative Crime Data}
Crimes known to law enforcement were gathered using incident-level administrative crime data released by the NYPD, the Los Angeles Police Department and the Chicago Police Department. Each city provides details about the location of a criminal incident, although the categories differ across cities. 

We use these data to determine whether a crime occurred inside a residence or in a public location.  In NYC, an incident is classified as residential if it occurred inside a private home, an apartment building or a public housing residence. In Los Angeles, we count as residential any crime in a residence, condomium/townhouse, foster home, dormitory, group home, shelter, mobile home, nursing home, transitional home, or yard. All other crimes are considered to have occurred in a public location. In Chicago, we count as residential any crime in a residence, apartment, basement, Chicago Housing Authority structure, college residence, hotel, residential driveway, or nursing home. All other crimes are considered to have occurred in a public location. 

\subsubsection{National Crime Victimization Survey}
Both the 2015-2019 and 2020 waves of the NCVS ask respondents to identify the type of location in which they were victimized. Using the 2015-2019 NCVS, we consider a crime to have occurred in a residential setting if the crime occurred at our near the victim's home or at or near a friend, neighbor or relative's house. Crimes that occurred in a commercial place, a parking lot, a school or another unspecified location are assumed to have occurred in a public location. 

The 2020 NCVS elicits more granular information on crime locations than the 2015-2019 NCVS.  We map the more granular 2020 location categories on to the less granular 2019 location categories as follows. Crimes that occurred in these locations are considered to be residential crimes: 1) in one's own dwelling, own attached garage or enclosed porch, 2) in a detached building on one's own property, 3) in a vacation/second home, 4) in a hotel or motel room that the respondent was staying in, 5) in one's own yard or driveway, 6) in a hallway, laundry room or storage area in one's own apartment building or 7) in a yard belonging to one's own apartment building. All other crimes are assumed to have occurred in a public location.

\subsection{Safegraph data}
We use the SafeGraph Neighborhood Patterns data \citep{sg_neighb} to calculate a count of people in each Census Block Group (CBG) every month. We build this off of the ``device home areas'' column, which gives a count of visitors by the visitors' home CBG. The home CBG is needed for weighting: we scale the number of visits from each CBG by the inverse of the sampling probability in that CBG, following \citet{van2020using} and \citet{sg_scale}. CBG populations are downloaded from SafeGraph's Open Census data, and the counts of devices in each CBG comes from SafeGraph's monthly Neighborhood Home Panel Summary files. The population of each CBG from Census, along with SafeGraph's reported device sample within each origin CBG, allows us to calculate the sampling probability each month.


\clearpage
\newpage 
%%%%%%%%%%%%%%%%%%%%%%%%%%%%%%%%%%%%%%%%%%%%%%%%%%%%%%%%%%%%%%
%%%%%%%%%%%%%%%%%%%%%%%%%%%%%%%%%%%%%%%%%%%%%%%%%%%%%%%%%%%%%%
%%%%%%%%%%%%%%%%%%%%%%%%%%%%%%%%%%%%%%%%%%%%%%%%%%%%%%%%%%%%%%
%%%%%%%%%%%%%%%%%%%%%%%%%%%%%%%%%%%%%%%%%%%%%%%%%%%%%%%%%%%%%%
\section{City-Specific Results} \label{sec:appendix2}
%%%%%%%%%%%%%%%%%%%%%%%%%%%%%%%%%%%%%%%%%%%%%%%%%%%%%%%%%%%%%%
%%%%%%%%%%%%%%%%%%%%%%%%%%%%%%%%%%%%%%%%%%%%%%%%%%%%%%%%%%%%%%
%%%%%%%%%%%%%%%%%%%%%%%%%%%%%%%%%%%%%%%%%%%%%%%%%%%%%%%%%%%%%%
%%%%%%%%%%%%%%%%%%%%%%%%%%%%%%%%%%%%%%%%%%%%%%%%%%%%%%%%%%%%%%
For brevity, in the main body of the paper, we present aggregated results for NYC, Los Angeles and Chicago as a whole. In this section, we present each set of results --- changes in mobility, reported crimes and street crime victimization risk --- for each of the three cities individually. While there is some variation in the evolution of crime rates and victimization risk among the three cities, the findings are substantively similar and suggest that some of the dynamics set in motion by the pandemic have been fairly universal.

In all three cities, mobility dropped considerably in March and April 2020. While foot traffic began to recover during the summer, each of the three cities ended the year with notably lower foot traffic than during the same months in 2019. Along with mobility, violent street crimes fell dramatically in April 2020 relative to April 2019 --- by 50\%, 20\% and 43\% --- in NYC, Los Angeles and Chicago, respectively. In all three cities, street crimes began to converge back to pre-pandemic levels over the summer.  However, by years' end, street violence remained approximately 10-15\% lower than it had been in 2019. Overall, summing across the March through December period,  public violence per 1,000 residents declined by 25\%, 7\% and 24\% in NYC, Los Angeles and Chicago, respectively.

With respect to public violence, all three cities experienced an upward shift in risk of victimization in 2020. However, the initial increase was the largest in Los Angeles where street crimes fell less sharply than in NYC or Chicago despite a sizable decline in outdoor activity. In Chicago, the change in risk was quite modest with risk increasing year-over-year by less than 5\% in most months. Despite the fact that each of the three cities faced a qualitatively different shift in victimization risk, in all three cities the risk of victimization and the crime rate diverged markedly in 2020. Overall, summing across the March through December period, the risk of public violence increased by 10\%, 23\% and 3\% in NYC, Los Angeles and Chicago, respectively.

\begin{figure}[h!]
   \caption{Violent crime and foot traffic in New York City}
     \hspace*{-1cm}
     \begin{tabular}{cc}
       \includegraphics[scale=0.6]{figs/nyc_effective_visitors1m.pdf}  & \includegraphics[scale=0.6]{figs/nyc_violentXoutside.pdf} \\ 
      (A) SafeGraph foot traffic &
       (B) Violent crime  \\
           \multicolumn{2}{c}{\includegraphics[scale=0.6]{figs/nyc_risk.pdf}} \\ 
     \multicolumn{2}{c}{(C) Risk} \\ 
    \end{tabular}
    \label{fig:chi_safegraph}
    \vspace*{-4mm}  \\ 
        \newline 
    Note: Figure plots the monthly change in foot traffic using Safegraph data (Panel A), the monthly number of non-residential (public) violent crimes (Panel B) and the monthly number of non-residential (public) violent crimes per 100,000 visitors based on mobility data from Safegraph (Panel C). Data are presented for NYC. In each plot, data are presented separately for 2019 (the blue line) and 2020 (the red line). 

\end{figure}

\clearpage
\begin{figure}[h!]
   \caption{Violent crime and foot traffic in Los Angeles}
     \hspace*{-1cm}
     \begin{tabular}{cc}
       \includegraphics[scale=0.6]{figs/la_effective_visitors1m.pdf}  & \includegraphics[scale=0.6]{figs/la_violentXoutside.pdf} \\ 
      (A) SafeGraph foot traffic &
       (B) Violent crime  \\
           \multicolumn{2}{c}{\includegraphics[scale=0.6]{figs/la_risk.pdf}} \\ 
     \multicolumn{2}{c}{(C) Risk} \\ 
    \end{tabular}
    \label{fig:chi_safegraph}
    \vspace*{-4mm}  \\  
        \newline 
    Note: Figure plots the monthly change in foot traffic using Safegraph data (Panel A), the monthly number of non-residential (public) violent crimes (Panel B) and the monthly number of non-residential (public) violent crimes per 100,000 visitors based on mobility data from Safegraph (Panel C). Data are presented for Los Angeles. In each plot, data are presented separately for 2019 (the blue line) and 2020 (the red line). 
\end{figure}

\clearpage
\begin{figure}[h!]
   \caption{Violent crime and foot traffic in Chicago}
     \hspace*{-1cm}
     \begin{tabular}{cc}
       \includegraphics[scale=0.6]{figs/chi_effective_visitors1m.pdf}  & \includegraphics[scale=0.6]{figs/chi_violentXoutside.pdf} \\ 
      (A) SafeGraph foot traffic &
       (B) Violent crime  \\
           \multicolumn{2}{c}{\includegraphics[scale=0.6]{figs/chi_risk.pdf}} \\ 
     \multicolumn{2}{c}{(C) Risk} \\ 
    \end{tabular}
    \label{fig:chi_safegraph}
    \vspace*{-4mm}  \\ 
        \newline 
    Note: Figure plots the monthly change in foot traffic using Safegraph data (Panel A), the monthly number of non-residential (public) violent crimes (Panel B) and the monthly number of non-residential (public) violent crimes per 100,000 visitors based on mobility data from Safegraph (Panel C). Data are presented for Chicago. In each plot, data are presented separately for 2019 (the blue line) and 2020 (the red line). 
 
\end{figure}



\newpage
\clearpage
%%%%%%%%%%%%%%%%%%%%%%%%%%%%%%%%%%%%%%%%%%%%%%%%%%%%%%%%%%%%%%
%%%%%%%%%%%%%%%%%%%%%%%%%%%%%%%%%%%%%%%%%%%%%%%%%%%%%%%%%%%%%%
%%%%%%%%%%%%%%%%%%%%%%%%%%%%%%%%%%%%%%%%%%%%%%%%%%%%%%%%%%%%%%
%%%%%%%%%%%%%%%%%%%%%%%%%%%%%%%%%%%%%%%%%%%%%%%%%%%%%%%%%%%%%%
\section{Robustness} \label{sec:appendix1}
%%%%%%%%%%%%%%%%%%%%%%%%%%%%%%%%%%%%%%%%%%%%%%%%%%%%%%%%%%%%%%
%%%%%%%%%%%%%%%%%%%%%%%%%%%%%%%%%%%%%%%%%%%%%%%%%%%%%%%%%%%%%%
%%%%%%%%%%%%%%%%%%%%%%%%%%%%%%%%%%%%%%%%%%%%%%%%%%%%%%%%%%%%%%
%%%%%%%%%%%%%%%%%%%%%%%%%%%%%%%%%%%%%%%%%%%%%%%%%%%%%%%%%%%%%%
In this section we provide alternative results using different methods to calculate victimization risk. First, we use survey data from the ATUS rather than Safegraph mobility data to activity-adjust the number of crimes. Next, we use a narrower definition of public spaces, excluding crimes committed indoors in public locations.


\subsection{Alternative Measure of Victimization Risk}
Our primary estimate of the risk of public violence is computed using Safegraph data which is the only source of mobility data which is publicly available prior to 2020. In this appendix, we re-compute our measure of victimization risk drawing on survey data from the American Time Use Survey (ATUS). To protect respondent anonymity, the survey data do not contain sufficient geographic detail to identify respondents living in NYC, Los Angeles and Chicago. We therefore focus on respondents who were living in metropolitan areas in 2019 and 2020. \autoref{fig:atus_risk} is identical to Panel B of \autoref{fig:city_crime_combined} except that ATUS data is substituted for Safegraph data in computing victimization risk. The $y$-axis represents the risk per 1 million hours spent outdoors.

  \begin{figure}
    \begin{center}
    \includegraphics[scale=1]{figs/atus_risk_combined.pdf}
    \caption{Risk in New York, Los Angeles, and Chicago using ATUS}
    \label{fig:atus_risk}
    \end{center}
            \vspace*{0mm}  \\ 
        \newline 
Note: Figure plots the number of non-residential (public) violent crimes per 1 million person-hours outside based on survey data from the American Time Use Survey (ATUS) and crime for the combined study sample (NYC $+$ Los Angeles $+$ Chicago). Data are presented separately for 2019 (the blue line) and 2020 (the red line). The ATUS was not administered in April 2020.
\end{figure}



As the ATUS was temporarily suspended from mid-March until mid-May 2020 due to public health concerns, we are not able to observe the change in victimization risk that occurred just after the beginning of the pandemic.\footnote{See: \url{https://www.bls.gov/tus/covid19.htm}.} However, for the remainder of the year, estimates of victimization risk using the Safegraph data and the ATUS survey data are substantively the same: that the risk of street crime victimization increased markedly in Spring 2020 and remained elevated --- by approximately 10\% --- during the remainder of the year.

 
\subsection{Alternative Measure of Public Locations}  \label{sec:alt_public}

We begin by plotting the change in the incidence and risk of public violence using an alternative definition of what it means for the crime to have occurred in public.  \autoref{fig:city_crime_alt_combined} is analogous to \autoref{fig:city_crime_combined} except it uses a more limited definition of public violence. In this figure, we count only crimes that occurred on streets, alleyways, public parks, or transit system. Estimated changes in risk are extraordinarily similar to the main estimates reported in the paper.

%\newpage 
%\clearpage
\begin{figure}[h!]
    \caption{Change in Public Violence and Victimization Risk using Alternative Definition of Public Crimes --- New York, Los Angeles and Chicago (2019-2020)}
     \begin{center}
     \begin{tabular}{c}
     \includegraphics[scale=0.6]{figs/crimes_alt_combined.pdf} \\ 
     (A) Violent crimes \\ 
     \includegraphics[scale=0.6]{figs/risk_alt_combined.pdf} \\ 
     (B) Risk, SafeGraph \\ 
    \end{tabular}
         \end{center}
    \label{fig:city_crime_alt_combined}
    \vspace*{0.5cm}  \\ 
        \newline 
    Note: This plot is analogous to \autoref{fig:city_crime_combined} except it uses a more limited definition of public violence. In each city, we count only crimes that occurred on streets, alleyways, public parks, or transit system. 
\end{figure}

\subsection{Changes in Outdoor Time Use}
Here, we address the possibility that the types of activities people engaged in while in public may have changed in 2020. To the extent that people spent more time engaged in activities that carry a higher degree of inherent risk, the increase in victimization risk that we observe could be an artifact of a shift in time use. In Appendix Figure \ref{fig:atus_where}, we plot the share of time away from home spent in retail establishments and restaurants, at work or school, in a vehicle, at a place of worship, and walking for the 2015-2020 period. While the overall amount of time spent outdoors in 2020 was lower than in previous years, there is no dramatic change in the mix of activities. 

Time spent walking was a slightly lower percent of time in public in 2020 (12\%) as opposed to 2019 (16\%). Time spent in businesses appears to increase sharply in 2020, although this may be an artifact of an anomalously low point estimate in 2019. The percent of time spent at workplaces or school, in a vehicle, or in place of worship were only slightly changed compared to 2019. The results align with \citet{jay2020neighbourhood}, using a different SafeGraph dataset, showing relatively uniform declines across places of interest. 

Importantly, the results show that time spent commuting was a smaller share of time in public in 2020. Since previous studies have found that time in transit has the highest risk of victimization \citep{vaughan2021promise, lemieux2012risk}, this suggests that activities in public were slightly less risky based on historical patterns. However, more granular data is needed to judge this conclusively.

\begin{figure}
\begin{center}
\includegraphics[scale=1]{figs/atus_where_change.pdf}
\caption{Change in outdoor time use}
\label{fig:atus_where}
\end{center}
        \vspace*{0mm} 
Note: This shows changes in where ATUS respondents spent their time away from home, using broad categories of their responses to the ``where'' question. We restrict to respondents living in urban areas in the months April-December. There were no reported visits to places of worship in 2018. \textbf{Categories:} Retail/Restaurant includes time spent at a restaurant or bar, grocery story, other store / mall, library, bank, gym, post office. Work or School is respondent's workplace or school. Transit includes cars, buses, subways, trains, bicycles, boats, and airplanes. Walking is only walking. Worship is place of worship.The Other category includes outdoors away from home and unspecified.'' 
\end{figure}

\newpage
\clearpage
%%%%%%%%%%%%%%%%%%%%%%%%%%%%%%%%%%%%%%%%%%%%%%%%%%%%%%%%%%%%%%
%%%%%%%%%%%%%%%%%%%%%%%%%%%%%%%%%%%%%%%%%%%%%%%%%%%%%%%%%%%%%%
%%%%%%%%%%%%%%%%%%%%%%%%%%%%%%%%%%%%%%%%%%%%%%%%%%%%%%%%%%%%%%
%%%%%%%%%%%%%%%%%%%%%%%%%%%%%%%%%%%%%%%%%%%%%%%%%%%%%%%%%%%%%%
\section{Statistical Inference} \label{sec:inference}
%%%%%%%%%%%%%%%%%%%%%%%%%%%%%%%%%%%%%%%%%%%%%%%%%%%%%%%%%%%%%%
%%%%%%%%%%%%%%%%%%%%%%%%%%%%%%%%%%%%%%%%%%%%%%%%%%%%%%%%%%%%%%
%%%%%%%%%%%%%%%%%%%%%%%%%%%%%%%%%%%%%%%%%%%%%%%%%%%%%%%%%%%%%%
%%%%%%%%%%%%%%%%%%%%%%%%%%%%%%%%%%%%%%%%%%%%%%%%%%%%%%%%%%%%%%
Our analysis studies the evolution of victimization risk in U.S. cities between 2019 and 2020.  The available evidence suggests that while the crime rate fell in 2020, the probability of being attacked in public rose. The data, however, are derived from a sample and so it is natural to wonder whether the observed difference could be an artifact of sampling variability. In this appendix we provide details for how we derive a sampling distribution for the comparisons we make in the main body of the paper.

Our core estimand is the change in risk between 2019 to 2020, where risk is measured as crimes divided by activity. In all analyses, the measures of crime and activity come from different data sources so standard parametric inferential procedures are unavailable. 
Instead we generate confidence intervals using a bootstrap procedure in which we sample 1,000 times from the underlying data. We perform two separate analysis --- one for our national-level analysis using survey data from the ATUS and the NCVS and one for our analysis of city-level crimes and Safegraph data.  We describe each of these procedures in further detail and present confidence intervals below.

\paragraph{National-Level Analysis Using Survey Data} The American Time Use Survey (ATUS) and National Crime Victimization Survey (NCVS) are both random samples of U.S. households. In the ATUS, we bootstrap sample at the household level, stratifying by month and year and the stratification variables used in the ATUS: race (Hispanic, Non-Hispanic Black, Non-Hispanic Nonblack) and four household types based on the age of residing children \citep{atus_user_guide}. In the NCVS, we bootstrap at the household level using the provided sample stratification variable. We re-sample 1,000 times with replacement, each time computing our key estimand: the percent change in risk, where risk is the number of victimizations per hour spent in a public space. This procedure captures variability in this statistic that is an artifact of sampling error.

\paragraph{City-Level Analysis Using SafeGraph data} The city-level crime data that we use \citep{nyc_crime_data, chicago_crime_data, la_crime_data} are not samples. These data provide a population-level measure of the numerator in our analysis: crimes known to law enforcement in each city. Hence these counts remain constant in the bootstrap procedure.

The SafeGraph data originates from mobile phone location data. Sampling variability arises from the fact that locations are recorded for only a sample of city visitors. We do not have access to the underlying individual-level data, so we cannot replicate the core sampling procedure. Instead, our bootstrap method re-sample from the U.S. Census Block Group $\times$ month observations provided in Safegraph's Neighborhood Patterns data \citep{sg_neighb}.  We thus simulate a scenario in which the data had covered a sample, drawn randomly with replacement, of \textit{neighborhoods}. Since the neighborhood is a much larger sampling unit than a person, this should arguably generate a more conservative confidence interval by increasing the variability in the estimates of the change in aggregate activity. \\

We present the results of this procedure in Appendix Table \ref{tab:atus_ncvs_table}. For the city-level analysis, we find that public victimization risk rose by 14\% in the combined NYC, Los Angeles and Chicago sample. The bootstrap 95\% confidence interval around this estimate ranges from 12.4\% to 15.7\%, indicating that a positive estimate is highly unlikely to be an artifact of sampling error. 

Next we turn to the national analysis using survey data. Here, our estimates are based on a nationally representative sample of American individuals and households. Our principal estimate is the risk public violence among residents of metropolitan areas. For this sample, we estimate that victimization risk rose by 24\% (row 4), although the 95\% confidence interval includes zero. As a check that to ensure that the finding is not an artifact of sampling variability, we  present estimates that toggle the restrictions used for the NCVS (outside, violent, and metro). We show estimates for all public crimes in the Unites States (row 1), all public violent crimes in the United States (row 2) and public crimes in metropolitan areas (row 3). In rows 1 and 2, all ATUS respondents are used instead of just those in urban areas to align with the NCVS sample. While the confidence intervals are wider using the survey data (which are sparser) and sometimes cross zero, the analyses taken together suggest that it is unlikely that any of these results are an artifact of sampling variability. $p$-values range from 0.01 to 0.13.



%We present the results of this procedure in Appendix Table \ref{tab:city_sg_table} (city-level analysis) and Appendix Table \ref{tab:atus_ncvs_table} (national analysis). For the city-level analysis, we find that public victimization risk rose by 11\% in NYC, 23\% in Los Angeles and 3\% in Chicago. For NYC and Los Angeles, these estimates are highly significant with confidence intervals ranging from 8\%-13\% for NYC and 21\%-25\% for Los Angeles. In Chicago, where the change in victimization risk was much smaller, the confidence interval (-1\% to +7\%) overlaps with zero. 

%Next we turn to the national analysis using survey data. Here, our estimates are based on a nationally representative sample of American individuals and households. Our principle estimate is for public violence among residents living in metropolitan areas. For this sample, we estimate that victimization risk rose by 24\%. We also present estimates for all public crimes in the Unites States (row 1), public violent crimes in the United States (row 2) and public crimes in metropolitan areas (row 3). While the confidence intervals are wider using the survey data (which are sparser) and sometimes cross zero, the evidence suggests that it is unlikely that any of these results are an artifact of sampling variability. $p$-values range from 0.01 to 0.08.

\newpage
\input{tables.tex}


\newpage
\clearpage
%%%%%%%%%%%%%%%%%%%%%%%%%%%%%%%%%%%%%%%%%%%%%%%%%%%%%%%%%%%%%%
%%%%%%%%%%%%%%%%%%%%%%%%%%%%%%%%%%%%%%%%%%%%%%%%%%%%%%%%%%%%%%
%%%%%%%%%%%%%%%%%%%%%%%%%%%%%%%%%%%%%%%%%%%%%%%%%%%%%%%%%%%%%%
%%%%%%%%%%%%%%%%%%%%%%%%%%%%%%%%%%%%%%%%%%%%%%%%%%%%%%%%%%%%%%
\section{Alternative Explanations} \label{sec:appendix_alternative}
%%%%%%%%%%%%%%%%%%%%%%%%%%%%%%%%%%%%%%%%%%%%%%%%%%%%%%%%%%%%%%
%%%%%%%%%%%%%%%%%%%%%%%%%%%%%%%%%%%%%%%%%%%%%%%%%%%%%%%%%%%%%%
%%%%%%%%%%%%%%%%%%%%%%%%%%%%%%%%%%%%%%%%%%%%%%%%%%%%%%%%%%%%%%
%%%%%%%%%%%%%%%%%%%%%%%%%%%%%%%%%%%%%%%%%%%%%%%%%%%%%%%%%%%%%%
In the main body of the paper, we observed that while the official crime rate was lower in 2020 than it had been in 2019, the risk of public violence increased. In this appendix, we provide evidence that this increase in risk is unlikely to be an artifact of changes in crime reporting behavior by victims or compositional shifts in the pool of people spending time outside.

\subsection{Victim Reporting} \label{sec:reporting}
To the extent that victims became more reluctant to report crimes to law enforcement due to public health risks, a legitimacy crisis in policing \citep{tyler2004enhancing, tankebe2014police, wolfe2016effect} sparked by the killing of George Floyd \citep{nix2021more}, or the perception that police and prosecutors were otherwise occupied, declining crime might be a mechanical artifact of a change in reporting behavior \citep{van1979victim, levitt1998relationship, davis2003willingness}. In using the official data and transforming it to account for changes in mobility, we might then understate the increase in victimization risk. Similarly, we could overstate the risk of crime if reporting increased.
\begin{figure}[h!]
     \begin{center}
    \caption{Share of Outdoor Violent Crimes Reported to Police --- 2015-2019 and 2020 National Crime Victimization Surveys}
    \includegraphics[scale=0.7]{figs/reporting_rate.pdf}
    \label{fig:reporting_rate}
     \end{center}
        \vspace*{-8mm}  
        \newline 
Note: Figure plots the percentage of non-residential (public) violent crimes reported to law enforcement in the 2015-2019 and 2020 waves of the National Crime Victimization Survey (NCVS), restricted to the third and fourth quarter surveys. For each survey wave, the point estimate is provided along with a 95\% confidence interval.
\end{figure}
To address this concern, we use an NCVS question on whether victims reported crimes to law enforcement. Appendix Figure \ref{fig:reporting_rate} presents the share of outdoor violent crimes that victims reported crimes to police among respondents interviewed in the 2015-2019 and 2020 waves of the survey. To maintain consistency with our administrative data from large cities, we focus on survey respondents living in metropolitan areas. Contrary to concerns that crime reporting may have fallen during the pandemic, the rate at which outdoor violent crimes were reported to the police was not unusual in 2020.


\subsection{Victim Selection} \label{sec:selection}
A second potential explanation is victim selection, the possibility that the composition of individuals who spent time outdoors and were therefore at risk of street crime victimization changed in 2020. Since the pandemic has had unequal public health impacts according to gender, race, and especially age \citep[e.g.][]{hutchins2020covid, miller2021estimated}, each of which is among the strongest predictors of victimization risk \citep[e.g.][]{perkins1997age}, this is a concern for someone who would like to understand how victimization risk changed for the typical American resident. In the extreme, it is possible that the aggregate risk of victimization could increase while decreasing for every demographic group, a case of Simpson's paradox \citep{blyth1972simpson}. In this section, we motivate three tests for the importance of victim selection using national data on time use from the ATUS, city-specific data on realized crime victimization, and within-neighborhood changes in victimization risk derived from both municipal microdata and Safegraph patterns data. None of the three tests suggests that the increase in victimization risk that we observe after March 2020 is likely to be an artifact of selection.

\subsubsection{Test of demographic selection using national survey data} \label{sec:demog_selection}
Our first test of selection uses a DFL-style reweighting exercise \citep{dinardo1996labor} to assess the contribution of selection into activity to the crime rate. This computation addresses how crime would have changed if activity had evolved as observed and victimization rates were fixed at their pre-2020 levels. 

Using data from the 2015-2019 NCVS we compute public victimization rates for each of eighteen age-race-gender cells defined by the intersection of race (White, Black, Other), gender (male, female) and age ($<$ 25, 26-49 and $>$ 50).%\footnote{We use less granular age categories to guard against drawing inferences from small bins in the ATUS data.}$^{,}$
%\footnote{In Appendix Figure \ref{fig:victimization_by_inc}, recognizing the importance of income, we replicate this analysis using age-race-income groups instead. Results are substantively similar.}  
These data are combined with estimates from the ATUS, which gives the share of time spent outside in 2019 and 2020 for each group. In \autoref{fig:victimization_by_group2} we plot baseline time use  (Panel A), the change in time use between 2019 and 2020 (Panel B) and baseline victimization rates (Panel C) for each of the eighteen demographic groups. 

Prior to the pandemic, though older people spent less time outdoors than younger people, there was little variation in time spent outdoors among the eighteen groups. How did time spent outside change after the COVID-19 pandemic? The decline in the share of time spent in public spaces was fairly universal and, across all groups, there was a 16\% decrease in waking hours spent outdoors in 2020. However, several dimensions of heterogeneity are notable. First, the behavioral response to the pandemic was notably muted among prime-age Black men. Second, the largest disruption to pre-pandemic behavior occurred in individuals under the age of 25, some of whose routine activities were disrupted by school closures. 

In Panel C, we consider hetereogeneity in baseline victimization risk, measured during the pre-pandemic period. In the figure, the blue bars represent the number of violent victimizations occurring in public locations per 1,000 people in each group. The red bars use the data from Panel A to compute activity-adjusted victimization risk. This computation shrinks or inflates each group’s victimization rate according its outdoor activity relative to the population average. For example, white males under 25 spend more time outside: 10.9 hours compared to an average of 9.6 hours in the ATUS population. Their activity-adjusted victimization rate is scaled by $9.6/10.9$ or about 0.88. Accounting for their increased activity makes their activity-adjusted victimization rate lower.

As is evident from the raw victimization rates, there is considerable heterogeneity in victimization rates across the eighteen groups with an especially strong age gradient. While young men face annual outdoor violent victimization rates of nearly 30 per 1,000, individuals over the age of 50 face rates that are uniformly below 10 per 1,000 and are sometimes as low as 5 per 1,000. For individuals below age 50, men face uniformly higher victimization rates than women with the gender gradient attenuating somewhat in the highest age category. When we adjust victimization rates for differences in time spent outside, the differences between groups narrow albeit only slightly. An implication of this result is that, prior to the pandemic, between-group differences in victimization risk are not well explained by between-group differences in mobility. 

Given the outsize importance of age in predicting victimization risk and the fact that time spent outside declined more among younger individuals than among older individuals, selection effects point to a \emph{reduction} in baseline risk in 2020. To formalize this observation, we predict the change in victimization risk that would arise from an observed shift in the demographics of individuals spending time outside their homes. We compute three quantities: (i) the total number of public violent crimes experienced by Americans in the pre-pandemic period, (ii) the expected change in victimization given the observed change in time spent in public for each demographic subgroup and (iii) the expected change in victimization assuming that every demographic subgroup reduced their time spent in public by the same amount in 2020. The difference between (ii) and (iii) indicates the importance of demographic selection effects. 

Below we motivate a formal model to account for the impact of compositional changes on victimization risk. We begin by noting that the annual number of outdoor crimes experienced by Americans in 2019 can be expressed as:
\begin{equation} \label{eq:selection1}
C_{2019} = \sum_{j \in J} n_{j,2019} \times ShareOutside_{j,2019} \times v_{j,2019}
\end{equation}
where $n_{j,2019}$ is the number of people in group $j$ in 2019, $ShareOutside_{j,2019}$ is the share of time individuals in group $j$ spent outside in 2019 and $v_{j,2019}$ is the group's outdoor victimization rate, defined as the number of outdoor crimes divided by the number of person-hours spent at risk:
\[
v_j = \frac{OutdoorCrimes_{j,2019}}{n_{j,2019}  \times ShareOutside_{j,2019}}
\]

Next we employ this notation to compute the two counterfactual quantities that we use to identify the importance of compositional effects. Let $\Delta_j$ be the change in the proportion of outdoor time, where $\Delta_j= ShareOutside_{j,2020} - ShareOutside_{j,2019}$.\footnote{In practice, to reduce noise, we use the 2015-2019 pre-period.} The number of crimes that would have been experienced by Americans, holding victimization rates constant but allowing behavioral responses to the pandemic (i.e., the change in the amount of time spent in public spaces) to vary by group is given by $\hat{C}_{2020}$:
\begin{equation} \label{eq:selection2}
\hat{C}_{2020} = \sum n_{j,2020} \times (ShareOutside_{j,2019} +\Delta_j) \times v_{j,2019}
\end{equation}
Equations \href{eq:selection1}{(\ref{eq:selection1})} and \href{eq:selection2}{(\ref{eq:selection2})} are identical except that, in \href{eq:selection2}{(\ref{eq:selection2})}, each group changes the share of time spent in public spaces in 2020 according to $\Delta_{j}$. $\hat{C}_{2020}$ is the predicted number of outdoor violent crimes in metropolitan areas of the United States, holding hourly victimization risk constant.

Next, we consider the counterfactual victimization rate under the constraint that all groups have an identical response to the pandemic (i.e., all groups reduce their time spent outdoors by the same quantity), substituting $\bar{\Delta}$, the population-weighted average across all $J$ groups, for $\Delta_j$:
\begin{equation} \label{eq:selection3}
C^*_{2020} = \sum n_{j,2020} \times (ShareOutside_{j,2019} + \bar{\Delta}) \times v_{j,2019}
\end{equation}
The number of outdoor violent crimes under this ``alternative pandemic", $C^*_{2020}$, is generated by constraining each group to have an identical response with respect to time use. The ratio of $\hat{C}_{2020}$ to $C_{2020}^{*}$ provides a measure of the importance of compositional effects---the additional number of crimes per day that we would expect, given constant victimization risk, from a heterogeneous as compared to a homogenous behavioral response to the pandemic. 

%We use the NCVS to obtain an estimate of $v_{i}$ for each of eighteen age-gender race groups (\autoref{fig:victimization_by_group2}) and the ATUS to estimate $\bar{\Delta}$ as well as $\Delta_{i}$ for each demographic subgroup. We begin by computing the total number of crimes experienced by Americans in the pre-pandemic period, $C_{2019}$. Taking a weighted sum over each of the groups, we obtain an annual estimate of 1.34 million outdoor violent crimes.\footnote{Formally $v_{i}$ is obtained by dividing the victimization rate for each group by the number of hours spent outside by individuals in each group during the six-month recall period in the NCVS.} Next, we compute $ \hat{C}_{2020}$ and $C_{2020}^{*}$. Using observed time use in 2020 and pre-determined victimization risk, we estimate that $\hat{C_{2020}}$ = 923,000 outdoor violent crimes. This number is 31\% smaller than the number of crimes measured in 2019 which is consistent with time spent outdoors having declined. Finally, constraining each group to have changed their public time use by $\bar{\Delta}$ = 15.8\%, we obtain an estimate of $C_{2020}^{*}$ = 970,000 outdoor violent crimes. The difference between $C_{2020}^{*}$ and $\hat{C}_{2020}$ is approximately 47,000 crimes, indicating that selection effects, captured using changes in the demography of individuals spending time outdoors, would predict a 5\% \textit{decrease} in crime. While the pandemic led to some re-sorting of individuals in outdoor spaces in 2020, on net, compositional changes, based on key observable dimensions of victimization risk, do not predict a large change in offending.



In order to generate estimates according to this framework, we begin by computing the total number of crimes experienced by Americans living in large metro areas in the pre-pandemic period. Summing over each of the eighteen groups, we obtain an annual estimate of 1.34 million outdoor violent crimes. Next, holding victimization risk fixed, we estimate that 923,000 outdoor violent crimes would have accrued on the basis of the observed reduction in time spent in public across the eighteen demographic groups. This number is 31\% less than the number of crimes in 2019, reflecting the reduction in time spent in public. Finally, constraining each group to have changed their public time use by 15.8\%, the mean decline in time spent outside across all groups, we obtain an estimate of 970,000 outdoor violent crimes. The difference between the counterfactual condition in (3) and counterfactual condition in (4) is approximately 47,000 crimes, indicating that demographic selection effects would predict a 5\% \textit{decrease} in crime. This is consistent with our casual observation that, if anything, the risk set grew relatively \emph{older} in 2020. %Notably, when we make the same adjustment using age-race-income groups, selection effects are effectively zero --- see Appendix Figure \ref{fig:victimization_by_inc}. 


\begin{figure}
     \begin{center}
    \caption{Victimizations per 1,000 Population for Selected Demographic Groups, 2015-2019 National Crime Victimization Survey}
    \begin{tabular}{c}
    \hspace*{-3cm}
    % \includegraphics[scale=.55]{figs/VictimizationATUSNCVSfig.pdf}
    \includegraphics[scale=.55]{figs/VictimizationATUSNCVSfig_ver2.pdf}
    \end{tabular}
    \label{fig:victimization_by_group2}
         \end{center}
        \vspace*{0mm}  \\ 
        \newline 
Note: Figure provides descriptive evidence on time use and victimization rates prior to the COVID-19 pandemic. Panel (A) summarizes the percentage of time spent away from one's home or yard using the 2015-2019 American Time Use Survey (ATUS) for each of eighteen age-gender-race groups.  Panel (B) summarizes non-residential (public) violent victimizations per 1,000 population using the 2015-2019 waves of the National Crime Victimization Survey (NCVS). The blue bars are the unadjusted rates. The purple bars are rates of public violent victimizations per person-hour spent in public.
\end{figure}


\newpage 
\subsubsection{Test of income-based selection using national survey data}
While age, race and gender are particularly salient predictors of victimization, socioeconomic status is also a noteworthy predictor. Given that higher income earners were more likely to be able to work from home during the COVID-19 pandemic, it is possible that the mix of people spending time outdoors in 2020 shifted towards lower-income workers. 

In order to test for whether any such shifts occurred and whether such a shift might compromise our estimates, we repeat the analysis presented in Section \ref{sec:demog_selection} for age-race-income groups where we divide the population into those earning above \$50,000 and those earning below \$50,000. In Panel A, we plot the number of hours spent outside by each of the 12 demographic groups. In Panel B, we plot the change in this share between 2019 and 2020. In Panel C, we compute activity-adjusted victimization rates for each group. 

Finally, as in our main analysis, we use this information to compute a DFL-style re-weighting estimate. The difference between $\hat{C}_{2020}$ and $C_{2020}^{*}$ implies that selection effects with these categories would predict a 0.1\% decrease in crime. Overall, while there was some re-sorting of time spent outside during the COVID-19 pandemic, this re-sorting does not predict higher crime victimization.  

\begin{figure}
     \begin{center}
    \caption{Victimizations per 1,000 Population for Selected Demographic Groups, 2015-2019 National Crime Victimization Survey}
    \begin{tabular}{c}
    \hspace*{-3cm}
    \includegraphics[scale=.55]{figs/VictimizationATUSNCVSfig_incomesplit.pdf}
    \end{tabular}
    \label{fig:victimization_by_inc}
         \end{center}
        \vspace*{0mm}  \\ 
        \newline 
Note: Figure provides descriptive evidence on time use and victimization rates prior to the COVID-19 pandemic, split by income. Panel (A) summarizes the percentage of time spent away from one's home or yard using the 2015-2019 American Time Use Survey (ATUS) for ge-income-race groups.  Panel (B) summarizes non-residential (public) violent victimizations per 1,000 population using the 2015-2019 waves of the National Crime Victimization Survey (NCVS). The blue bars are the unadjusted rates. The maroon bars adjust for hours spent in public.
\end{figure}

\newpage 
\subsubsection{Direct test based on victim demographics in NYC and Los Angeles} \label{selection2}

%Finally, we assess the presence of demographic selection effects directly using crime microdata from NYC and Los Angeles, where data on the demography of crime victims are available. If our finding of an increase in risk is driven by selection, crime victims in 2020 should, in general, be those who have higher baseline victimization risk than in the past. Evidence from this analysis is presented in Appendix \ref{selection2} and suggests that changes in the demography of potential crime victims is unlikely to account for a meaningful share of the large increase in victimization risk that we observe in the data. Consistent with the above analysis, 2020 crime victims had lower, not higher, historical rates of victimization.

We can also assess the size of potential selection effects directly using crime microdata from NYC and Los Angeles. If our finding of an increase in risk is driven by selection, crime victims in 2020 should, in general, be those who have higher baseline victimization risk than in the past. We can probe this possibility using data on victim demographics. As in the main results, we restrict to violent crimes that occurred outdoors.

We match each victim to their expected victimization rate according to their age, race, and gender based on data from the 2015-2019 National Crime Victimization Surveys (NCVS), restricting to outdoor crimes and to respondents living in cities with populations over one million. We use 24 demographic cells in total, representing the interaction of three race/ethnicity groups (White, Black, Hispanic), two genders (male, female), and four age groups (12-18, 18-24, 25-44, 45+).\footnote{The last age group is an imperfect match with the NCVS, which bins all ages 35-49. Victims 45+ in the NYC and LA crime data are matched with NCVS respondents over 50.} More formally, the NCVS data yields $v_g$, a victimization rate for group $g$, with $g \in \{ 1,...,24\}$ where, e.g., $v_1$ could be the victimization rate White females under 18 which is 20 per 1,000 (per year). Next, we assign the victimization rate $v_g$ to each crime in the NYC and LA data according to victim demographics and calculate the average victimization rate for each quarter $q$:
\begin{equation}
\overline{v}_q = \frac{ \sum_{g=1}^{24} v_g * N_{g,q}}{N_q}
\end{equation}
where  $N_{g,q}$ is the total number of victims who belong to group $g$ in quarter $q$ and $N_q$ is the total count of crimes in quarter $q$ where we observe victim demographics. The time series of $\overline{v}_q$, the average expected victimization rate, shows us how selection into crime victimization based on risk changes over time.

Using data for the combined two-city sample, we plot the average expected victimization rate of victims over time in \autoref{fig:historical_avg_vic_rates}. The series drops sharply in the second quarter of 2020 and remains lower, decreasing from 15.1 to 14.7 violent offenses per 1,000 people across years. This suggests that, if anything, crime victims of 2020 were \emph{less likely} to have been victimized than in previous years. This implies that within-group increases in risk were likely larger than in our aggregated estimates. Taken together, the available evidence suggests that compositional changes in the demography of potential crime victims is unlikely to account for a meaningful share of the large increase in victimization risk that we observe in the data.

\begin{figure}[h!]
    \caption{Average historical victimization rates}
    \label{fig:historical_avg_vic_rates}
    \begin{center}
    \includegraphics{figs/historical_avg_vic_rates.pdf}
    \end{center}
\end{figure}


%In this appendix we provide additional results on the demographic composition of crime victims in NYC and Los Angeles prior to and during the COVID-19 pandemic. We also provide descriptive evidence on the time-path of disturbances to public health as the timing of the impact of the COVID-19 pandemic differed among the three cities in our sample. 

%\autoref{fig:victim_demog} shows additional details on victim demographics, drawing from microdata in NYC and Los Angeles which include information on the age, race and gender of crime victims. Across the two cities, changes in the age distribution of victims in 2020 were  modest. There was a small decrease in the relative shares of victims under the age of 18 or between the ages of 18 and 24. As these are the age groups that are most likely to be victimized at baseline, if anything, it appears as though changes in the demography of victims point to reduced risk of victimization rather than an increase in victimization risk.

\begin{comment}
\begin{figure}[h!]
\begin{center}
    \caption{Victim demographics, New York and Los Angeles}     \label{fig:victim_demog}
    %  \hspace*{-.6cm}
     \begin{tabular}{cc}
       \includegraphics[scale=0.55]{figs/vic_ages_male_minus.pdf}  & \includegraphics[scale=0.55]{figs/vic_ages_female_minus.pdf} \\ 
       (A) Age, male victims &
       (B) Age, female victims   \\
       \includegraphics[scale=0.55]{figs/vic_races_male_minus.pdf}  & \includegraphics[scale=0.55]{figs/vic_races_female_minus.pdf} \\ 
       (C) Race, male victims &
       (D) Race, female victims    \\
    \end{tabular}
    \end{center}
    \vspace*{0.5cm}  \\ 
    \small
    Note: These plots show how the composition of violent crime victims changed in New York City and Los Angeles, the two cities where victim demographics are available. 
\end{figure}


\begin{figure}
\begin{center}
    \caption{Historical victimization rates of victims, New York City and Los Angeles}
    \includegraphics[scale=0.7]{figs/historical_avg_vic_rates.pdf}
    \label{fig:historical_avg_vic_rates}
\end{center}
        \vspace*{-4mm}  \\ 
        \newline 
    Note: Figure presents average historical victimization rates of crime victims from the NYPD and LAPD incident data. The historical victimization rates are calculated using data from the 2015-2019 waves of the NCVS and victim demographics are reported in the incident data. Let $g$ index demographic groups and $v_g$ be the victimization rate for group $g$. The average historical victimization rate in quarter $q$ is given by:
\[
\overline{v}_q = \frac{ \sum_{g} v_g * N_{g,q}}{N_q}
\]
where $N_{g,q}$ is crimes against group $g$ in quarter $q$ and $N_q$ is total crimes.
\end{figure}
\end{comment}

\subsubsection{Test of geographic selection}
Given wide variation in crime rates across different communities, another potential source of selection is geographic. If time spent outdoors changed more in some communities than in others---for example, if individuals living in higher poverty communities were less able to shelter at home---changes in risk could be an artifact of between-community differences in victimization risk. We assess whether the estimates presented in Figure 2 attenuate when we focus on the change in victimization \emph{within a given community}. This analysis is especially salient as Census block group fixed effects explain nearly 80\% of the variation in crime risk among our three cities.

We study within-community changes in the following specification:
\begin{equation}
Risk_{cmy} = \sum_{k \in \{1,...,12\}} \beta_k 1\{m=k, y=2020\} + \alpha_{cm} + \epsilon_{cmy} 
\end{equation}
where $Risk_{cmy}$ gives risk per 100,000 visitors in census block group $c$ during month $m$ in year $y$. $\alpha_{cm}$ denote Census block group by month fixed effects, so that the coefficients $\beta_k$ track the within-area change in risk compared to the same month in 2019. Observations are weighted by the number of visitors, and standard errors are clustered at the Census block group level. 

\begin{figure}
     \begin{center}
    \caption{Victimization Risk, Conditional on Census Block Group Fixed Effects}
    \begin{tabular}{c}
     \includegraphics[scale=1]{figs/cbg_within_coefs.pdf}    \\
 
    \end{tabular}
    \label{fig:awesome}
         \end{center}
        \vspace*{0mm}  \\ 
        \newline 
Note: Figure plots coefficients on month-of-year dummy variables  from a regression of victimization risk on month, year and Census block group fixed effects. The horizontal line at $y$=0 indicates no change in victimization risk between 2019 and 2020, conditional on Census block group fixed effects. The 95\% confidence interval is plotted using standard errors clustered at the Census block group level.
\end{figure}


We plot coefficients $\beta_k$ and the associated 95\% confidence intervals in \autoref{fig:awesome}. The pattern in the coefficients is similar to those reported in \autoref{fig:city_crime_combined}, indicating that victimization risk rose most steeply in April and May 2020 but remained elevated throughout the remainder of the year. This suggests that the increase in risk that we observe is not driven by a shift in the geographic composition of the population at risk. 



\newpage
\clearpage
%%%%%%%%%%%%%%%%%%%%%%%%%%%%%%%%%%%%%%%%%%%%%%%%%%%%%%%%%%%%%%
%%%%%%%%%%%%%%%%%%%%%%%%%%%%%%%%%%%%%%%%%%%%%%%%%%%%%%%%%%%%%%
%%%%%%%%%%%%%%%%%%%%%%%%%%%%%%%%%%%%%%%%%%%%%%%%%%%%%%%%%%%%%%
%%%%%%%%%%%%%%%%%%%%%%%%%%%%%%%%%%%%%%%%%%%%%%%%%%%%%%%%%%%%%%
\section{Heterogeneity in the Change in Risk} \label{sec:heterogeneity}
%%%%%%%%%%%%%%%%%%%%%%%%%%%%%%%%%%%%%%%%%%%%%%%%%%%%%%%%%%%%%%
%%%%%%%%%%%%%%%%%%%%%%%%%%%%%%%%%%%%%%%%%%%%%%%%%%%%%%%%%%%%%%
%%%%%%%%%%%%%%%%%%%%%%%%%%%%%%%%%%%%%%%%%%%%%%%%%%%%%%%%%%%%%%
%%%%%%%%%%%%%%%%%%%%%%%%%%%%%%%%%%%%%%%%%%%%%%%%%%%%%%%%%%%%%%
In 2020, the risk of public crime victimization rose in U.S. cities. A natural next step is to consider whether the changes were heterogeneous. In other words, did some demographic subgroups disproportionately bear the burden of increased crime risk? Given that risk rose between as well as within neighborhoods, the scope for heterogeneous treatment effects is narrower than an unconditional analysis would allow. Nevertheless, using national survey data from the ATUS and the NCVS, we compute the change in victimization risk by age ($<$35, $>$35), gender and race (White, Black, Other). These estimates are presented in Appendix Table \ref{tab:atus_ncvs_demog_table}, along with 95\% confidence intervals calculated as in \autoref{sec:inference}.  Unfortunately, the effectively small samples mean that all changes are measured with a high degree of uncertainty.  Given the width of the confidence intervals, we are unable to make any strong claims about which groups experienced an outsize change in victimization risk.

%\newpage 
\clearpage
\begin{table}[]
    \caption{Change in risk for demographic groups, NCVS and ATUS}
    \label{tab:atus_ncvs_demog_table}
    \begin{center}
    \begin{tabular}{lcc}
    Demographic & Percent change in risk & 95\% CI \\ \hline
    \input{figs/atus_ncvs_demog_table}
    \hline 
    \end{tabular}
    \end{center}
    \textit{Note:} This table shows the percent change in risk for select demographic groups using the ATUS and NCVS, restricing to respondents in metro areas (ATUS and NCVS) and to violent offenses outside (NCVS). Confidence intervals are calculated using the same bootstrap procedure described in \autoref{sec:inference}. 
\end{table}

